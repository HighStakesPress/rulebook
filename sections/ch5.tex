\documentclass[../main.tex]{subfiles}

%----------------------------------------------------------------------------------------
%	PART MECHANICS
%----------------------------------------------------------------------------------------



\part{Playing The Game}


%----------------------------------------------------------------------------------------
%	CHAPTER  RULES
%----------------------------------------------------------------------------------------

\chapter{Basic Rules}


\begin{displayquote}
\textit{The judge smiled. Men are born for games. Nothing else. Every child knows that play is nobler than work. He knows too that the worth or merit of a game is not inherent in the game itself but rather in the value of that which is put at hazard. Games of chance require a wager to have meaning at all.}
\end{displayquote}

\rightline{{\rm --- Cormac McCarthy, \textit{Blood Meridian, or the Evening Redness in the West}}}

\hfill

\lettrine[lines=2]{E}{very} game has rules, and American Sunset is no different. Part II is devoted to the rules of play. The rules here encompass (nearly) every situation you will encounter while playing. We recommend that both the Game Master (GM) and players be familiar with this section before playing. We have also included here optional rules, to allow for slight changes to certain mechanics so you may customize the game you want to play.

Keep in mind that for all the details of the system in this chapter, the GM has the final say on all rulings.

\section{Skills}\label{Skills}

Skills represent how your character is capable of interacting with the world. Though a character's background and Reputation define who they are, their skills represent what they can do. All player characters (and most non-player characters) have an index of skills rated from 0 (lowest) to 10 (highest) that measure their talents at various tasks. The full list of skills is as follows:

\begin{multicols}{2}

\begin{description}
\item [] Awareness
\item [] Deception
\item [] Education
\item [] Empathy
\item [] Fighting
\item [] Fitness
\item [] Grit
\item [] Handiness
\item [] Medical
\end{description}

\begin{description}
\item [] Perform
\item [] Poise
\item [] Intimidate
\item [] Riding
\item [] Rope Skills
\item [] Shooting
\item [] Sleight of Hand
\item [] Stealth
\item [] Survival
\end{description}



\end{multicols}

\subsection{Skill Descriptions}
All skills will have their time to shine at some point during gameplay, but it may not always be immediately obvious what each skill is best suited for what task. For this reason, a full description of each skill is listed here.

\begin{multicols}{2}


\paragraph{Awareness}
Awareness is your sensitivity to the environment around you, and dictates how effective  your senses and  intuition are. The higher your awareness is, the more likely you are to quickly foresee changes in the social or physical environment, and the more likely you will be able to react to them. Awareness governs your ability to notice things as well as your Initiative. (See Turn Actions, pg. \pageref{Turn Actions})

\paragraph{Deception}
Deception is your skill at lying. Characters with a high Deception rating are skilled at lying through their teeth, telling impromptu fables, and otherwise tricking people with their words. 

\paragraph{Education}
Education is how well you can read and write, how much you know about academic subjects, and your skill at logical debate. A character with an Education rating of 0 is illiterate. A high Education skill allows you to read more complex texts and reference them from memory. Additionally, Education determines your skill bonus for Logical Rhetorical Strategies.  (See Rhetorical Strategies, pg. \pageref{Rhetorical Strategies})

\paragraph{Empathy}
Empathy is your ability to understand other people, especially their emotions, and a higher rating in  this skill allows you to motivate others more effectively. Your Empathy can be used  for social and narrative purposes, and to treat Strain. Empathy determines your skill bonus for Empathetic Rhetorical Strategies. (See Rhetorical Strategies, pg. \pageref{Rhetorical Strategies})

\paragraph{Fighting}
Fighting represents your abilities in hand to hand combat, be it brawling with your fists, using weapons like knives and sabers, or other improvised weapons like chairs, tools, or other found objects. Your Fighting rating determines your skill bonus using melee weapons.  (See Weapons, pg. \pageref{Weapons})

\paragraph{Fitness}
Fitness represents your physical prowess. How far you can run, how fast you can run, how much you can lift, etc. A high Fitness rating allows you to effectively perform improvised actions, such as breaking down doors or hopping onto a moving train. Successful Fitness contests allow additional movement actions in combat. (See Movement, pg. \pageref{Movement})

\paragraph{Grit}
Grit is your ability to endure pain and cope with injuries you receive. The more Grit you have, the more Wounds you can take before becoming incapacitatd. Grit can also be used to Evade a weapon attack, instead of Counter-Attacking. (See Initiating a Contest, pg. \pageref{Initiating a Contest})

\paragraph{Handiness}
Handiness is your ability to perform trade crafts and to make, fix, and operate objects with mechanical complexity, such as locks or engines. Handiness is used for narrative purposes to repair complicated machines, jerry-rig small mechanisms, or operate a powered vehicle such as a train or steamboat.

\paragraph{Intimidate}
Intimidate is your ability to successfully threaten other people. A character with a high Intimidate rating is good at frightening others and coercing them to surrender. Your Intimidate skill may also be called upon for narrative purposes. Intimidate determines your skill bonus for Hostile Rhetorical Strategies. (See Rhetorical Strategies, pg. \pageref{Rhetorical Strategies})

\paragraph{Medical}
Medical is your knowledge of medical practice, etiquette and first aid. This skill is used to treat Wounds people receive on the battlefield. Using the Medical skill  often requires tools, and a lack of access to these tools, such as bandages, disinfectant, or surgery equipment may hamper the use of this skill. (See Treating Wounds, pg. \pageref{Treating Wounds})

\paragraph{Perform}
Perform is your ability to create artistic illusions, play musical instruments or sing, and to maintain a disguise. For 

\paragraph{Poise}
Poise is your ability to maintain your composure through the strain you may receive. The more Poise you have, the more Strain boxes are available to you. (See Wounds and Strain, pg. \pageref{Wounds and Strain}) Poise can also be used to try and avoid taking Strain in response to an enemy's social attack on you, instead of attacking back. 

\paragraph{Riding}
Riding represents your ability to ride a horse and to do things while mounted. A Riding Contest is one of the contests that is taken during any Travel Contest made on horseback. (See Travel Contests, pg. \pageref{Travel Contests}) A character can also use a horse to Trample an unmounted opponent. Rules regarding horses in combat can be found in Mounted Combat, see pg. \pageref{Mounted Combat}.

\paragraph{Rope Skill}
Rope Skill represents your ability to use rope, be it to tie knots well or quickly, do rope tricks, or lasso somebody. Rope skill determines your skill bonus while using a Lasso, in combat or otherwise.  (See Weapons, pg. \pageref{Weapons})

\paragraph{Shooting}
Shooting represents your ability to shoot a gun or a bow. Shooting determines your skill bonus when using most ranged weapons. (See Weapons, pg. \pageref{Weapons})

\paragraph{Sleight of Hand}
Your Sleight of Hand rating dictates your skill cheating in games of chance, committing theft, and other acts of skullduggery and dexterity. This includes your skill bonus with thrown weapons. (See Weapons, pg. \pageref{Weapons})

\paragraph{Stealth}
Stealth is your ability to move around without making others aware of your presence. When hiding, your Stealth skill is tested against the other's Awareness to determine if they notice you. Characters that don't notice you cannot Counter-Attack in the first round of combat. (See Ambush Round, pg. \pageref{Ambush Round})

\paragraph{Survival}
Survival is your ability to feed yourself, identify flora and fauna, track both people and animals, and find shelter in the wild. You will be called upon to use Survival when tracking anything or making any kind of Travel Contest. (See Travel Contests, pg. \pageref{Travel Contests})


\end{multicols}

\subsection{Skill Ratings}
Skill Ratings represent a character's ability to perform a specific set of tasks on a scale from zero to ten. For a rough understanding of what a specific skill rating means as it relates to a  talent, consider this table:

\begin{description}
\item[0:] Entirely Untrained. May have an understanding of what this skill pertains to, but has no practical experience with it.
\item[1:] Barely Trained. Has attempted the skill several times, enough to understand the basics.
\item[2:] Inadequately Trained
\item[3:] Adequately Trained, Competent
\item[4:] Above Average, Proficient
\item[5:] Professional Quality. For practical skills, you could reasonably expect to make a career out of your talents at this level.
\item[6:] High Quality Professional, Above Average Professional, Noted Amongst Professionals
\item[7:] Expert, Exemplar of Your Profession, A Cult Phenomenon
\item[8:] Master, A Popular Phenomenon, History is Waiting for You to Prove Yourself
\item[9:] Aspiring Legend, One of the Best.
\item[10:]  Legendary, Your Name is a Quality Unto Itself. The Stradivarius, Newton, or Alexander the Great of your craft.
\end{description}

\section{Contests}\label{Contests}
Contests are dramatic devices used to resolve conflicts between two or more characters or between a character and their environment. Contests are dramatic and decisive, designed to produce a clear winner and loser (with the occasional draw). All contests begin by determining what is at stake, and all contests end by deferring the stakes to the winner. This system, in which there is always something being wagered, is the primary innovation of American Sunset. There will be dozens of contests in a single session, so be sure that you understand contests before proceeding any farther in the book.

\subsection{Contest Types}

There are 3 different types of Contests: Opposed, Unopposed, and Extended.

\paragraph{Opposed Contests}\label{Opposed Contests}
Contests in which two or more characters compete against each other to fulfill their own goals. When an Opposed Contest is initiated, the acting character is called the attacker and the inactive character is called the defender. The attacker and defender set the stakes and each roll 2d6. Both attacker and defender add their skill bonus to their roll, respectively, to determine their total score. The character with the higher total is the winner of the contest, and the character with the lower total is the loser.  

Depending upon the situation, there may be multiple attackers and/or defenders (see Multiple Participants in Combat, pg. \pageref{Multiple Participants in Combat}). 

Examples of Opposed Contests include combat, sneaking by another character unnoticed, or any other contest in which one character actively resists another.  

\paragraph{Unopposed Contests}
Contests in which an active character employs a skill against something that doesn't actively resist. In these tests, the GM assigns a Target Number which the player's total must exceed in order to succeed in the contest. Examples include scaling a mountain, shooting a tin can off a fence, or breaking down a door.

\paragraph{Extended Contests}
Contests in which multiple smaller contests, opposed or otherwise, are resolved for a larger set of stakes. These contests are usually time sensitive, and will typically follow structured time (resolving actions in terms of rounds and seconds) instead of narrative time (resolving actions in terms of whoever speaks first, etc.)

Extended Contests are not initiated like normal contests. In an Extended Contest, multiple contests are made in fulfillment of a larger goal. These larger goals are called Conditions, and dictate the winners and losers of Extended Contests. At the beginning of an Extended Contest, the stakes are determined in addition to the conditions in which they  are applied. Note that only Extended Contests have Conditions.

Multiple Contests will likely take place over the course of an Extended Contest, as characters vie to fulfill the prescribed Conditions for their own benefit. The stakes are resolved only when the Conditions have been met. 

\textit{Example: After a quick exchange of unpleasantries, Jessica's Gang fights The Wild Thirteen for control of Timmy's Tavern. The opponents set the stakes: the last gang remaining in the tavern wins. The Stakes are determined: the victor gains control of the bar and if the loser is The Wild Thirteen, then they may never return to Silver Springs. Both parties proceed into combat aware of the stakes and what they need to do in order to secure them.}

Extended Contests are designed to streamline the stakes system, allowing GMs to set larger stakes in which multiple, less significant contests are required. Additionally, Traveling is a specific type of Extended Contest with its own set of rules. 
 
\subsection{Initiating a Contest}\label{Initiating a Contest}
 
Whenever a Contest is initiated, there is a five step process involved in resolving it:

\begin{enumerate}
\item \textbf{Determine the Stakes}
\item \textbf{Raise the Stakes}
\item \textbf{Make Rolls}
\item \textbf{Calculate Totals}
\item \textbf{Resolve the Stakes}
\end{enumerate}

The details of each step are explained here.

\hfill

\begin{enumerate}
	\item \textbf{Determine the Stakes.} Before any dice are rolled, the GM and the players determine the stakes. The active character (i.e. the player whose turn it is) sets the stakes, and the GM must confirm that the stakes are appropriate for the situation. Examples of stakes include:
		\begin{itemize}
			\item \textbf{Injuries (Wounds and Strain):} Contests which occur during combat involve one side trying to harm the other, be it their pride (Strain) or their flesh (Wounds). Characters setting an injury as the stakes also pick the degree of injury they wish to inflict, choosing light, medium, or severe. The winner of the contest will inflict injury of the specified type upon the loser (see Combat, pg. {Combat}).

			\item \textbf{Physical Objects:} Money, Booze, Territory.

			\item \textbf{Narrative Advantage:} The favor of a crowd, intimidation of an opponent, demonstration of skill to a mentor.
		\end{itemize}

	\item \textbf{Raise the Stakes.} If the defending character is a Reputable Character (i.e. they have a Reputation Rating), they may choose to take a penalty (-2 to their combat roll) to raise the stakes. Examples of raised stakes include: 
		\begin{itemize}
			\item \textbf{Increase the degree of an injury} being dealt by one (From Light to Medium or from Medium to Severe)

			\item \textbf{Add a Narrative Stake.} Establish something the defending character has to gain by winning this combat. \textit{Example: \enquote{If I shoot this bandit he'll drop the detonator for the dynamite.}}

		\end{itemize}

			\item \textbf{Defending.} In Combat, it is important to define the defender's reaction to the attacker. Defenders may either Counter-Attack, in which they attempt to instead inflict the Injury stake upon the attacker and resolve any other stakes in their favor, or they may Evade the attack in an attempt to nullify the stakes altogether. In a Counter-Attack, the defending character selects the weapon or Rhetorical Strategy they will use for the Counter-Attack (exactly like the attacker) to oppose the attacker. When a Defender Evades, they roll their Grit (vs. Weapons) or Poise (vs. Rhetorical Strategies) to oppose the attacker. (See Combat pg. \pageref{Combat}) \label{Defending}
		
	\item \textbf{Make Rolls.} Any characters participating in the contest roll an appropriate amount of dice: 2d6 normally, or 4d6 if that player is at Advantage or Disadvantage. When rolling with two dice, both die are totaled to determine the result. Once the result of the roll is determined, proceed onto Step 4.
		\begin{itemize}
			\item Characters rolling with Advantage roll four dice instead of two, and total the two highest die for the result of their roll.

			\item Characters rolling with Disadvantage roll four dice instead of two, totaling the two lowest die for the result of their roll.

			\item Characters with both Advantage and Disadvantage or with multiple instances of Advantage or Disadvantage use whichever they have more instances of, with no additional benefits for any instance beyond the first. If they have the same amount, they will roll as if they have neither. \textit{Example: A character who has three instances of Advantage and one instance of Disadvantage rolls as if they only had one instance of Advantage. A character with two instances of Advantage and two instances of Disadvantage rolls as if they had neither Advantage or Disadvantage.}

			\item It is also worth noting that the resolution system in steps 3 and 4 is designed to be modular. It is entirely possible to use a different method of choosing a winner and loser than rolling dice and adding skill ratings. If you would like to resolve the outcome of an in-game Texas Hold'em match by playing a round of Texas Hold'em, feel free! If you are using an alternate resolution system, however, you should specify this during Step 1. 
		\end{itemize}
		
	\item \textbf{Calculate Totals.} Once all characters have completed Step 3, they each add their designated Skill Rating and any other modifiers to the number they rolled. This value is compared to the Target Number (Unopposed) or their opponent's result (Opposed). 
		\begin{itemize}
			\item Total = 2d6 + Skill Rating + Other Modifiers (Qualities, Injuries, Landscapes etc.)
			\item In an Unopposed Contest, the character wins the contests if they exceed the Target Number. In an Opposed Contest, the character with the higher result is the winner and the character with the lower result loses. In the event of a draw in an Opposed Contest, any injuries at stake will be dealt to both characters and the remaining stakes are contested with exceptions to be determined by the GM.
		\end{itemize}
 	\item \textbf{Resolve the Stakes}
 		\begin{itemize}
			\item Any injuries are dealt and the effects of any other stakes are resolved before future contests are initiated.
 		\end{itemize}
\end{enumerate}

 
 
\subsection{Draws}\label{Draws}
In the event of a contest where both attacker and defender have the same total, a draw occurs. Whenever this happens, both players suffer the stakes that were set out for the contest. In the case of combat, any injuries at stake are dealt to both players. How exactly this occurs is up to the GM; it may be as simple as both bullets finding their mark, or environmental hazards causing other, unexpected effects. Outside of combat, consequences of a draw are less explicit. The core idea behind every draw, however, should be that both characters get what they want, and both characters get what they don't want. You should expect that all draws have some kind of broader narrative consequences as well; after all, a tie is a fateful event, and fateful events don't happen every day.


\subsection{Target Numbers}\label{Target Numbers}

During Opposed contests, determining the outcome is as easy as comparing the dice rolls of two characters. In Unopposed contests, however, it is up to the GM to determine the difficulty of the task at hand. All contests range in difficulty from approximately 4 to 22, though exceptionally simple or near-impossible tasks may be lower or higher, respectively. Below is a table showing some examples of target numbers for Awareness and Handiness contests.


\begin{tcolorbox}[ title= Example Target Numbers]
\begin{tabular}{ m{3cm} | m{4.5cm} | m{4.5cm}}
Target Number & Awareness Example & Handiness Example \\ \hline

 4 - Very Easy	& \footnotesize Realizing that the building you are in is on fire. 					& \footnotesize  Repairing a horseshoe with suitable equipment \\  \hline

 6 - Easy		& \footnotesize Finding a soured apple in shallow barrel of fresh apples.			& \footnotesize Tailoring a garment. \\  
			& \footnotesize Noticing someone who is obviously trailing you.					& \footnotesize Brewing low quality moonshine. \\ \hline	
			
 8 - Routine	& \footnotesize Noticing that your drinking buddy has not touched their drink.		& \footnotesize Making a rope from scratch. \\
 			& \footnotesize Keep watch for the night. 									& \footnotesize Forging an iron pot. \\  \hline
	
 10 - Challenging	& \footnotesize Spotting a moving figure on the horizon.					& \footnotesize Repairing a simplistic firearm with the Unreliable quality.\\
				& \footnotesize Realize when someone is Blathering at you. 				&\footnotesize Sewing a simple garment.\\
				& \footnotesize Eavesdrop on a conversation. 							&\footnotesize Repairing a broken sword or other melee weapon. \\  \hline	
	
 12 - Difficult		& \footnotesize Noticing that the bartender has a shotgun concealed behind the bar. 	& \footnotesize Crafting a sturdy bow. \\ 
 				& \footnotesize Eavesdrop on a conversation through a wooden door.		& \footnotesize  Crafting a simple bear trap.\\
				& \footnotesize Analyze a simple crime scene. 							& \footnotesize Crafting a simple melee weapon. \\ \hline
	
 14 - Hard 	& \footnotesize Distinguish a familiar individual at a Medium range. 				&\footnotesize Building a simple firearm. \\
 			& \footnotesize Overhear someone whispering.								&\footnotesize Repairing a complex firearm with the Unreliable quality. \\ \hline
	
 16 - Very Hard		& \footnotesize Realize that someone is secretly alluding to something.		&\footnotesize Crafting a simple lock.\\  
				& \footnotesize Notice a secret trap or trapdoor.							&\footnotesize Crafting a simple firearm.\\
				& \footnotesize Analyze a complex crime scene.						&\footnotesize Crafting a saddle.\\ \hline
	
 18 - Arduous 		& \footnotesize Read someone's lips. 								&\footnotesize Crafting an ornate melee weapon. \\
 				& \footnotesize Distinguish a familiar individual at Long Range. 				&\footnotesize Crafting a complex experimental firearm with the Unreliable quality.\\  \hline
	
 20 - \footnotesize Near Impossible	& \footnotesize Find a well-hidden secret trap or trapdoor. 		&\footnotesize  Crafting a complex lock.\\ \hline	
	
 22 - Legendary 	& \footnotesize Pursue a dead end and catch the culprit regardless.			&\footnotesize Crafting an item with Legendary craftsmanship. \\ \hline
 
 24 - Impossible		& \footnotesize Discern that someone is a killer by looking into their eyes. & \footnotesize Crafting a complex or experimental firearm without the Unreliable quality \\  
			 

\end{tabular}
\end{tcolorbox}

\subsection{Travel Contests}\label{Travel Contests}
\textbf{Traveling} is a unique type of Unopposed Contest, performed whenever players must travel long distances across the vast reaches of the frontier. Three different contests determine the success of a traveller: an Awareness Contest, either a Fitness or Riding Contest, depending on whether the character is riding a mount or not, and a Survival contest. Each contest is performed separately, with their own target numbers and consequences for failure. Different landscapes will have different degrees of difficulty for each skill, and may be easier or more difficult depending on whether or not they are mounted. For examples of travel difficulties, see \textbf{Landscapes} pg. \pageref{Landscapes}.

\textit{LANDSCAPES AND TRAVEL RULES TO BE EXPANDED}

\section{Reputation Points}\label{Reputation Points}
Traits, Feats, and Flaws make up a character's Reputation Aspects, but without Reputation Points, they are little more than personal details. By spending Reputation Points in game, characters can influence the outcomes of contests. The more points that are spent, the greater the influence, and the grander the effect the player has on the world. It is suggested that characters begin a campaign with a Reputation Rating of 3 and with 1 Reputation Point available to spend, but the GM can modify this if they wish. (See Reputation in Creating a Character, pg. \pageref{Creating a Character})

\subsection{Spending Reputation Points}
To spend a Reputation Point, a character must invoke one of their Aspects in a positive manner. A character can invoke an Aspect whenever that Aspect would benefit them in their course of action. (For more details, see Reputation, pg. \pageref{Reputation}).

This is not a strictly defined system, and the use of reputation should always be negotiated between the GM and the players. Players should invoke a Reputation Aspect at narratively dramatic moments, in which their character performs a heroic or otherwise significant act related to their Aspect. Keep in mind that when players have access to reputation they can choose to initiate contests they otherwise couldn't risk, knowing that they have their reputation to back them up.

Only one Reputation Point may be spent invoking an Occupation or Background Trait. Any number of Reputation Points may be spent invoking Feats.

\hfill

\begin{tcolorbox}[ title= Spending Reputation Points]
\begin{tabular}{ m{2cm} | p{10.5cm} }
Points Spent & Effects \\\hline


 1	& Gain Advantage on a single contest. \\  
	& \textit{Banjo Tom gains Advantage when playing his banjo.} \\ \hline
	
	
 2	& Gain Advantage on a single contest, regardless of how many instances of Disadvantage that character has. \\  
	& Automatically succeed at an Unopposed Contest with a Target Number that character is capable of reaching. \\
	& \textit{Banjo Tom automatically succeeds at playing a complicated song [TN 12]. } \\ \hline
	
	
 3	& Automatically succeed at an Unopposed Contest with a Target Number up to 5 beyond what that character is capable of reaching. \\
 	& \textit{Banjo Tom automatically succeeds at coherently playing a song in a rowdy tavern [TN 17].} \\  \hline
	
	
 4	& Gain Advantage for all contests in an Extended Contest. \\
	& \textit{Banjo Tom gains advantage when playing his banjo in an Extended Contest against his nemesis, Ukulele Dave.} \\  \hline
	
	
 5 	& Automatically succeed at any Unopposed Contest \\
 	& \textit{Banjo Tom plays his banjo while standing atop the tallest mountain. The village below the mountain can hear him perfectly, and are moved to tears by his soulful art.} \\ 
 

\end{tabular}
\end{tcolorbox}

\subsection{Gaining Reputation Points}

To gain Reputation Points, a character must invoke one of their Aspects in a negative manner. A character can invoke an Aspect to gain Reputation Points whenever that Aspect could  disadvantage the character. Invoking an Aspect to gain Reputation Points should alter the story in some way, and deviate the story from its natural direction. It is generally not considered in good form to have party members gaining Reputation Points by narratively hindering each other, and GMs should shy away from this. However, this dynamic could be useful in situations where inter-party conflict is present, and there are disastrous consequences if party cohesion is not maintained. 
GMs should consider giving multiple Reputation Points for especially tragic or disadvantageous invocations of an Aspect. 

\subsection{Reputation Tips}
When negotiating Reputation Points with your GM, it is important to remember a few things:

\begin{itemize}

\item They want you to use your traits. Your GM wants your character to live up to their reputation, so help them do so. Avoid choosing traits that are too narrow in their scope. Good traits are both descriptive and contribute to the GM's storytelling. 

\item Be active. Your GM is busy with other things, and may not be aware of your traits. Moreover, be aware of your other party members' traits. While only the GM can compel a Flaw, anybody can make a suggestion, especially you.

\item Be descriptive and unique. Your Reputation is not a matter to be taken lightly. It is the sum total of all the actions that people know you by, and your name should impress everyone. Your GM will be more likely to reward well-narrated invocations, especially when it could earn you another trait.
\end{itemize}