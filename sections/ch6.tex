\documentclass[../main.tex]{subfiles}

%----------------------------------------------------------------------------------------
%	CHAPTER  CHARACTER CREATION
%----------------------------------------------------------------------------------------

\chapter{Creating a Character}\label{Creating a Character}

\lettrine[lines=2]{P}{layer} Characters in American Sunset are unique among their peers, distinguished by their ability and willingness to build and wield a reputation. The average person in the world of American Sunset isn't seeking reputation and will often avoid garnering it if possible. Your character, however, shouldn't be like that. That's not to say you can't make your character a shopkeeper or have some other mundane job, but you should ensure that they have greater ambitions than simply running the shop.
There are five primary steps to generating a character: 

\begin{enumerate}
\item Creating a character concept 

\item Choosing a Background (pg. \pageref{Choosing a Background})

\item Choosing an Archetype (pg. \pageref{Choosing an Archetype})

\item Building Reputation (pg. \pageref{Reputation})

\item Choosing your starting gear (pg. \pageref{Starting Gear}). 
\end{enumerate}

All characters start as an idea, but it is their Background, Archetype, and Reputation that give that character a mechanical presence in the world of American Sunset. Descriptions of each of these steps can be found below.

\section{Character Concept}
Before you create your character, it's important to come up with a concept of who you want your character is and what your character  does. All characters start as an idea. Where did your character come from? What do they do for a living? How did they find themselves in the American frontier, and what do they hope to find there? Why are they a \enquote{somebody} out here in the West, rather than just another average resident of the frontier?

\begin{itemize}
\item Take time to develop a concept.
\item Consider your character's appearance and ethnicity. Who are their parents? How do they dress?
\item Consider your character's ambitions. What do they want to accomplish, who do they want to become?
\item Consider your character's skills and unique abilities. How do these things reflect who they are?
\end{itemize}

\section{Choosing a Character Background}\label{Choosing a Background}
The next stage of character creation is to choose a background. Backgrounds are divided into broad and distinct archetypes, and it is your job to define your character's background in detail. Your character's background will determine the initial skill ratings that you can allocate to your character's skills, and will give your character an additional bonus or penalty. These skill ratings are broken up into two categories: Archetype Skills and General Skills. Your archetype skills are determined by the archetype that you select during the next step, so keep that in mind! General skills can be allocated to any skills. It is recommended that you also select your Archetype before allocating skills.


\begin{multicols}{2}


\paragraph{No One Knows}

No one knows, but you certainly do. Are you running from something? Is that something following? If so, how closely? Nobody knows where you came from, but here you are. Do you plan on telling them the truth? For someone without a history, even legends and lies are a kind of truth. Your character's background should be very detailed. It should include the most important reason: why nobody else knows. Or perhaps there are those who do know, but they are so remote and distant that they are inconsequential.
Ensure that your GM knows your background prior to the beginning of the game. 

\textbf{Skills} 
\begin{itemize}
\item When you allocate your Archetype skills, allocate one skill at rating 6, two skills at rating 4, and three skills at rating 3. 
\item When allocating general skills allocate three skills at rating 2, and seven skills at rating 1.
\end{itemize}
\textbf{Background Abilities}
\begin{itemize}
\item Contests to find characters with this background roll at disadvantage.
\item At the beginning of each session, characters with this background may roll a d6. If they roll 1, 2 or 3 then the player may choose a different background trait. If they roll a 4, 5 or 6, then the GM chooses a new background trait for them.
\end{itemize}

\hfill

\textit{Example: Doctor Wieland's accent is a rarity in the West, and information on him is even rarer. Dressed in a tattered priest's garb, Wieland acts little like a man of god, and spends his days wandering between scattered settlements. Unbeknownst to most, \enquote{Doctor} Wieland is actually Gustav Wieland, the son of a missionary who immigrated to the continent many years ago. There are only two notes on Gustav in the government record: a mention of his and his father's efforts to convert a native tribe on the eastern shore, and his father's certificate of death --- suicide. Fleeing their failure, Wieland wanders the desert and fears that the curse laid upon him by the tribe's shaman- to never know God and to spend his days in exile- may have already come true.}


\paragraph{I Served My Time}

Your character has lived a long time, too long even. Whatever you did in your years past, it has shaped you dramatically. The skills you've developed over the years have been honed to a fine point? but you have never known the diversity which life offers to the dilettante. You are a professional at what you do, first and foremost. Examples of characters with this background  would include career soldiers, retirees, and criminals.

\textbf{Skills} 
\begin{itemize}
\item When you allocate your Archetype skills, allocate three skills at rating 5 and three skills at Rating 3. 
\item When allocating general skills, allocate three skills at rating 2 and four skills at rating 1.
\end{itemize}
\textbf{Background Ability}
\begin{itemize}
\item As a tradeoff for having more skills available to them, characters with this background begin with a permanent Light Strain box marked.
\end{itemize}

\hfill

\textit{Example: Luke \enquote{Lucky Luke} Baxter's an old, scarred man with windswept pepper-grey hair and a lifetime of experience. Lucky is man who's been around. A delinquent until his mid twenties, he was in and out of jail for most of his early life. He tried his hand at mining, lumber, and trapping before enlisting in the Army, fighting Indians across the growing frontier. Later serving in the Union Army during the Civil War, Luke has seen more combat than most men alive. Having walked away from so many injuries and disastrous battles, his nickname is as well-earned as any of his wartime accolades. During the war, his shrapnel-scarred countenance became widely known amongst Union soldiers, who considered him a good-luck charm. Go into battle alongside Lucky Luke and you're sure to survive, even if the battle isn't won.}

\textit{Lucky married his trade, and it's been so long that he's had trouble reintegrating into society and regular working life. Despite his relative youth and his issues with social life after the war, he's been respected and honored by many for his service. Recent attempts to settle down haven't gone down well, and Lucky has yet to find the right girl or the right place. To pay for his travels, he does odd jobs from patching roofs to hunting criminals. At the end of the day, when he's exhausted and can feel his growing age, he still finds himself hoping that he'll stumble upon a life of comfort, and find the girl of his dreams waiting there for him.}


\paragraph{On a Mission}

Who you are is indistinguishable from what you want to do. You set out into the world in pursuit of someone or something, and that pursuit is what drives you even when others would falter and fail. Perhaps you feel the need to spread your ideology to the world. Perhaps you're out to kill the man who killed someone you loved. Perhaps you're out to resolve some inner conflict within yourself. You live your life in a frenetic haze of passion, constantly judging people and events through the lens of this mission, be it holy or otherwise. If you're a religious character, consider which religion your character might follow. Unless they're a charlatan, their religion should have a profound impact on their mindset. 

\textbf{Skills} 
\begin{itemize}
\item When you allocate your Archetype skills, allocate one skill at rating 5, two skills at Rating 4, and three skills at rating 3. 
\item When allocating your general skills, allocate four skills at rating 2 and five skills at rating 1. 
\end{itemize}
\textbf{Background Ability}
\begin{itemize}
\item The first time in a session that a Character with this Background invokes their Background Trait, they do not have to spend a Reputation Point to invoke it.
\end{itemize}

\hfill

\textit{Example: Adam Migisi is a Native American of the Confederacy of the Hawk who is on a mission to claim the Confederacy of the Hawk for his own. He is afraid that the Starhawk Kenu will lead the Confederacy into an unremarkable death. He is afraid that his people will slowly fade away and will only be known as the savages that robbed a handful of trains, instead of the proud people who lost a great war against General Steele, but nonetheless killed the man who caused them so much suffering. Migisi served as a Union soldier in the Civil War, taking the name Adam on his papers.}

\textit{Migisi knows war now, better than even the warlords. If anyone is to lead the Confederacy of the Hawk into their final  war, it should be him. Every action Migisi takes, he takes to either attack General Steele, or to test himself in order to prove that he is more worthy to be the Starhawk than Kenu. He brings in the Sheriff's bounties constantly, testing himself against larger and larger groups of men. Whenever he works with others, he immediately takes a leadership role, ordering then and treating them like soldiers. If he is asked to perform a task that is beneath him, he will delegate it immediately. His reputation is beginning to spread, and his only concern is that his name will reach the ears of General Steele or the Starhawk Kenu before he is ready.}


\paragraph{A Life of Privilege}

You grew up in a rich and powerful household. While you may not have been exorbitantly wealthy, scarcity was rarely, if ever, an issue. Perhaps you were a member of the prosperous state bureaucracy, or the child of a tycoon lord. Yet, for some reason, you are out west - in the empty desert. Your reputation and your security are behind you now. These are the badlands, and they don't care about you unless you do something for them. It's an economy where your life is the nexus - made fragile for the first time.

\textbf{Skills} 
\begin{itemize}
\item When you allocate your Archetype skills, allocate one skill at rating 5, two skills at rating 4, and three skills at rating 3. 
\item When you allocate your general skills, allocate three skills at rating 2 and eight skills at rating 1.
\end{itemize}
\textbf{Background Ability}
\begin{itemize}
\item Instead of beginning the game with \$2d6, you begin the game with \$2d6+70.
\end{itemize}

\hfill

\textit{Example: Dulcinea Rivera lived a pampered life as the daughter of Mexican aristocrat Guillermo Rivera until Dulcinea mother was assassinated and they fled Mexico. After discovering her father's plans to have her marry into a wealthy family in Texas, she ran away from home, bringing with her as much money as she could carry. Dulcinea is not accustomed to being told she can't have things exactly the way she wants them, but she's well-educated, strong willed and proving herself to be resourceful on the frontier.}

\textit{Now on the run from bounty hunters hired by her father to bring her home, Dulcinea is constantly on the move. She's dedicated to proving that even without her father and his money she can become someone of note, and she strives to make something grand of herself. Perpetually one step ahead of her father's lackeys, she's developing a network of contacts to build up her own legacy.}



\paragraph{Outcast}

Whether it was exile or your choice, you never fit in where you were raised. Instead, you took to the West --- a land perfect for those who were misfits amongst their families. Nomadic life isn't new for you, and it's possible you derive some kind of stability from it. Even if you aren't a loner by nature, circumstances have made it your lot in life. The trick is to always be arriving and departing; sentimental relationships will only keep you from moving forward. 

There are a thousand and one reasons why you might have been cast out of your home, be they ethnicity, ideology, or even bad luck. In the wake of the Civil War, despite . While pride and enlightenment ideals are championed, equality doesn't happen overnight, and thousands of people like you have fallen through the cracks. 

\textbf{Skills} 
\begin{itemize}
\item When you  allocate your Archetype skills, allocate one skill at rating 5, two skills at rating 4, and three skills at rating 3.
\item When you allocate your general skills, allocate four skills at rating 2 and six skills at rating 1. 
\end{itemize}
\textbf{Background Ability}
\begin{itemize}
\item Characters with this Background gain a bonus to Travel Contests equal to their Reputation rating.
\end{itemize}

\hfill

\textit{Example: Adi Morrison was until recently a well meaning and well behaved Enochite with only the best interests of her family and faith at heart. She lived in a compound that, against the wishes of the head church, still practiced polygamy. Fearing retribution from the Federal Government as well as a higher power, the church elders deemed it necessary to send a Destroying Angel to dismantle the troublesome community.}

\textit{Near-mythical in their capabilities and often dismissed as little more than anti-Enochite rhetoric, Destroying Angels rarely leave witnesses to their deeds. Adi escaped the burning of her home only by a stroke of luck, fleeing across the desert knowing her husband, sister-wives, and children were already dead.}

\textit{Traveling to other Enochite settlements she found that they had all been forewarned of survivors from her group of \enquote{heathens}. Tainted by the rumor that she was responsible for the massacre at her home, Adi was refused entrance to compound after compound as, to her dismay, the story travelled much faster than she could. Eventually finding herself in Silver Springs, she finds little love amongst her former peers, but has earned respect among many who harbor hatred towards the Enochites. Scraping by as a seamstress, she spends her days searching for a way to prove that her story is the truth, though she doubts she'll ever return to the Church anytime soon.}

\paragraph{A Life of Struggle}

You've had a difficult life so far, and it's only going to get worse. Your days have been spent in grueling labor, bordering on starvation, or burdened by intense and pronounced stressors. You've likely seen the worst that humanity has to offer, and offered it yourself in turn. A life of suffering perpetuates suffering, often without intending to do so.
Now you're out in the west, and your life of struggle has prepared you well. Going without water for a day is nothing new to you. Killing someone for food maybe isn't too farfetched.  While you're inured against commonplace atrocities, the grotesque frontier beckons. Will you participate in the depravity, or work to abolish the suffering you know best?

\textbf{Skills} 
\begin{itemize}
\item When you  allocate your Archetype skills, allocate one skill at rating 5, two skills at rating 4, and three skills at rating 3.
\item When you allocate your general skills, allocate allocate six skills at rating 2 and two skills at rating 1.
\end{itemize}
\textbf{Background Ability}
\begin{itemize}
\item You begin with \$d6 instead of \$2d6
\end{itemize}

\hfill

\textit{Example: Philip Zande is an ex-slave from Georgia who has been jumping trains and boats since he escaped just before the Civil War, trying to make his way West. Word in his home was that there was a land of opportunity out West, far from white slave owners, where people took your real measure as a man. The youngest and healthiest of his family, he was chosen as the one to escape. Skipping meals for the next three days, his parents and siblings bundled their rations and gave them to a shocked Phillip, telling him to run, run and never stop until Jefferson. }

\textit{Now five years older than he was then, Phillip has survived by stealing money and food, and learned to defend himself in fights from other vagrants living the same life as him. Philip has spent his whole life only looking as far ahead as tomorrow, and that's prepared him much more than most. Now that he's made it to Jefferson, Phillip is learning to adapt to city life, and worked a number of labor jobs before being offered a position as a bodyguard for Summit Bank. Now stably employed for the first time in his life, his only goal now is to find some way to free his family and bring them to his new home in Silver Springs.}



\paragraph{Baptism by Fire}

Your life was ordinary until, one day, something big happened --- something that changed your life forever. It was at that moment when, upon reflection, you knew that you couldn't return to your ordinary life. Perhaps you were a bureaucrat who learned too much about a banker's schemes, or a train conductor whose vessel was sacked by marauding natives. Regardless of how it happened, you found yourself out west. You're unprepared for what's going to happen next, but you know that you can't go back.

\textbf{Skills} 
\begin{itemize}
\item When you  allocate your Archetype skills, allocate one skill at rating 5, two skills at rating 4, and three skills at rating 3.
\item When you allocate your general skills, allocate four skills at rating 2 and four skills at rating 1.
\end{itemize}
\textbf{Background Ability}
\begin{itemize}
\item Characters with this Background get a +1 when using rhetorical strategies against unreputable characters.
\end{itemize}



\hfill

\textit{Example: Amanda Cobart was a passenger on a train traveling from Chicago to San Francisco when it slowed to a grinding halt just hours outside of Silver Springs. Some suspected engine trouble, but it didn't take long for the whispers to reach to her compartment: train robbers. Always in the wrong place at the wrong time, the man sitting across the aisle from her was carrying two thousand shares in JDSF Railway in his suitcase --- exactly what the bandits were after. When the first robber made his way to her compartment, she took her opportunity to act. A two-shot derringer, tucked in her purse by a concerned family member, suddenly became more than just the small insurance against muggings it was intended to be. In two shots, the bandit was dead. Though she'd never fired a gun before , the slain bandit's revolver fit into Amanda's hands better than she could explain. As the other robbers rushed into her compartment, she killed each of them in turn, singlehandedly destroying one of the most notorious gangs in the area.}
\textit{Hailed as a hero, Amanda stepped off the train in Silver Springs a different woman than she was just hours ago, but her troubles were only just beginning.}

\end{multicols}


\section{Choosing an Archetype}\label{Choosing an Archetype}
Your character's Archetype determines their Archetype skills, and provides guidance towards choosing a Career Trait. There are six Archetypes to choose from: \textbf{Killer, Ranger, Educated Folk, Working Folk, Tradespeople, and Scum}. Every Player Character in American Sunset is defined as one of these categories. 

Included with each Background are two sets of skill ratings, one for Archetype skills and one for general skills. At character creation, you choose what skills your character will have by allocating the skill ratings listed in your Background to your character. To do this, use the values listed under your chosen Background. 

\textit{Example: Alice wants to make a Ranger with a background of I Served My Time. Because of the Archetype skills that I Served My Time provides, she gives her character a rating of 5 in Grit, Riding, and Fitness, and then she gives her character a rating of 3 in Handiness, Shooting, and Stealth. Next she chooses general skill ratings, giving her character a rating of 2 in Awareness, Fighting, and Sleight of Hand, and a rating of 1 in Education, Medical, Poise, and Intimidate.}

Archetype skills must be chosen from among the skills in your chosen archetype, but your general skill ratings can be put into any skill you choose (including any unallocated Archetype skills). See Skills (pg. \pageref{Skills}) for the full list of skills and more a detailed description of their uses. The Archetypes are as follows:

\begin{multicols}{2}


\paragraph{\underline{Killers}}

Killers, by trade or necessity, make their living by removing other people from the earth - or sometimes by merely threatening to kill them. Whether it be lawful and justified, in cold blood, or anywhere in between, Killers have the skills to live up to their name. Examples of Killer characters include bandits, soldiers, hired mercenaries, bounty hunters, or lawmen.

\begin{description}
\item[Archetype Skills]
\item[] Awareness
\item[] Fighting
\item[] Fitness
\item[] Grit
\item[] Intimidate
\item[] Medical
\item[] Riding
\item[] Shooting
\item[] Survival
\end{description}

\paragraph{\underline{Rangers}}

Rangers, don't just survive in the wilds, they thrive. Men and women who make their living on the land, Rangers are skilled at living for weeks, months, or indefinitely without support from the greater forces of civilization. Examples of Rangers include characters who are hunters, trappers, cowboys, trackers, or prospectors.

\begin{description}
\item[Archetype Skills]
\item[]Awareness
\item[]Fitness
\item[]Grit
\item[]Handiness
\item[]Rope Skill
\item[]Riding
\item[]Stealth
\item[]Shooting
\item[]Survival
\end{description}

\paragraph{\underline{Educated Folk}}

As an educated individual, you learned your skills from a school or university. Unlike other Archetypes, all educated folk can read as a benefit of their upbringing, and their technical abilities are often rare and highly valued on the frontier. Examples of Educated characters include bankers, judges, scholars, military officers, priests or doctors.

\begin{description}
\item[Archetype Skills]
\item[]Awareness
\item[]Deception
\item[]Education
\item[]Empathy
\item[]Handiness
\item[]Medical
\item[]Perform
\item[]Poise
\item[]Shooting
\end{description}



\paragraph{\underline{Working Folk}}

Working folk are the backbone of continued civilization on the frontier. No part of their life is easy, and not all chose it voluntarily. Strong-backed and stalwart, their jobs require stamina and long hours, often in unsafe conditions. Working Folk is a good choice for players who want to play as ranchers, loggers, laborers, farmers or miners.

\begin{description}
\item[Archetype Skills]
\item[]Awareness
\item[]Fitness
\item[]Fighting
\item[]Grit
\item[]Handiness
\item[]Intimidate
\item[]Rope Skill
\item[]Riding
\item[]Survival
\end{description}


\paragraph{\underline{Tradespeople}}

Tradespeople are working folk who've learned a craft, whether it was self taught or by an apprenticeship. While not formally educated, many of their talents cannot be learned without long hours of practice, a teacher, or both. Tradespeople includes careers such as railroad engineers, craftspeople, shopkeepers, coach operators, and other townsfolk.

\begin{description}
\item[Archetype Skills]
\item[]Awareness
\item[]Deception
\item[]Empathy
\item[]Fitness
\item[]Handiness
\item[]Poise
\item[]Perform
\item[]Riding
\item[]Shooting
\end{description}



\paragraph{\underline{Scum}}

Scum are the strange, the criminal, and the outcasts of the west. Scum are shunned by even the lowest classes of society, but still find a place in the lawless reaches of the West. Players should consider the Scum Archetype are those who want to play as drifters, prostitutes, con-men, gamblers, and mystics.

\begin{description}
\item[Archetype Skills]
\item[]Awareness
\item[]Deception
\item[]Education
\item[]Empathy
\item[]Fighting
\item[]Intimidate
\item[]Poise
\item[]Sleight of Hand
\item[]Stealth
\end{description}

\end{multicols}


\section{Reputation}\label{Reputation}

Your reputation influences many aspects of American Sunset. It's what makes your character stand out, and gives them an identity beyond their stats and skills. As your character develops a grander reputation, the more powerful a force they'll become.

\textbf{All characters begin with a Reputation Rating of 3 and one Reputation Point, but the GM may change this as they see fit.}

\subsection{Defining Aspects: Traits, Feats, and Flaws}

The fundamental building blocks of the Reputation system are your character's Aspects. There are three types of Aspect, Traits, Feats, and Flaws, which represent individual facets of your Reputation and the stories people tell about your character. All characters start with one Feat, one Flaw, and a Background and a Career Trait.

\textbf{If you cannot think of a quality Trait, Feat, or Flaw during Character Creation, feel free to define them during play.} If you do not have all of your Aspects completed prior to play, and your character responds to an event in a cool or dramatic way, make it a Feat! Just keep in mind though, that you cannot invoke an Aspect which does not exist. Be sure to inform your GM about any changes you make to your character sheet in this way.

\paragraph{Background Trait}
The first trait you will define is your Background Trait. Your Background Trait describes what brought you from being nobody in the West to being somebody of repute. It should elaborate on the Background you chose, solidifying it into a story. Mechanically, you are limited to spending only a single reputation when you invoke your background trait. (See Using Reputation Points, pg. \pageref{Reputation Points})

\textit{Example: Jessica James is a Gunslinger whose Background is No One Knows. She would write \enquote{No One Knows} in the Background Trait section. Jessica James comes from Missouri, and is unknown in Jefferson. Since her identification papers along with most of her other personal belongings were burned when a train caught on fire, nobody knows who she is, and she has used this to her advantage by spreading her reputation as a mysterious bounty hunter.}

\paragraph{Career Trait}
The second trait you should define is your Career Trait. It doesn't ask who your character is, but rather what they do for a living. While this may seem mundane, Reputation is about how people perceive your character and so includes details like your Career. Note that your character's Archetype is separate from your Career. While a character's Archetype determines their starting skill ratings, your Career Trait is what your character does on a day to day basis. Mechanically, the Career Trait will not allow you to spend more than a single point of Reputation to perform tasks you would be expected to perform in your chosen way of life, so keep that in mind while choosing which aspects of your character are best represented here (See Using Reputation Points, pg. \pageref{Reputation Points}).

\paragraph{Examples of Career Traits}

\begin{multicols}{2}

\begin{tcolorbox}
Career Trait: Farmhand 
\tcblower

Description: My character has worked on a farm from when he was very young, moving and using farm equipment, and tending to animals.  
\end{tcolorbox}

\begin{tcolorbox}
Career Trait: Nurse 
\tcblower

Description: My character is a nurse at the local hospital, where she tends to the sick and wounded.
\end{tcolorbox}

\end{multicols}

\paragraph{Feats}

Feats are predominantly positive attributes of your character that describe what they've done and what they're known for. Feats are made up of two parts: The feat itself and, its story. A good Feat is a specific one. Why is this something that makes your character stand out? What caused people to know this about your character? What does this feat say about your character's personality? Your feat should be more than \enquote{my character is known for being really strong!} or \enquote{my character is known for being very compassionate.} Plenty of people in the world are strong or compassionate. Why is your character that way?
 
Feats aren't just a single word that describes your character. Instead, they tell a story about how your character became who they are. Not only does this give you the opportunity to craft a more detailed backstory for your character, it will also help you determine when to invoke your reputation (See Using Reputation Points, pg. \pageref{Reputation Points}).

\paragraph{Defining Feats during Play}
When a character has an empty Feat slot and the character performs a remarkable action, the player may define one of their empty Feat slots immediately. Describe the action as actively as possible in the \enquote{Feat} entry, and be sure to tell a convincing story too. 


\paragraph{Examples of well-written Feats}

\begin{multicols}{2}

\begin{tcolorbox}
Feat: Strong during fits of grief  
\tcblower
Story: When my character's best friend was shot, he carried him on his back for fifteen miles into town to see a doctor. 
\end{tcolorbox}

\begin{tcolorbox}
Feat: Can treat any combat wound, no matter how gruesome 
\tcblower
Story: My character treated soldiers during the war when she was a combat medic, and knows her way around a bullet wound.
\end{tcolorbox}

\end{multicols}

\paragraph{Examples of poorly-written Feats (Don't write traits like this!)}

\begin{multicols}{2}


\begin{tcolorbox}
Feat: Really strong
\tcblower
Story: My character lifted lots of weights
\end{tcolorbox}

\begin{tcolorbox}
Feat: Good doctor
\tcblower
Story: My character is good at healing
\end{tcolorbox}

\end{multicols}

\paragraph{Flaws}
Flaws, like traits, are made of two parts. The flaw itself, and the weakness that governs the flaw. Unlike traits and feats, flaws are predominantly negative aspects of a character that prevent them from accomplishing their goals. 

You should ask yourself similar questions about your character's flaw as you ask about their traits. What about your character's flaw is unique? Under what circumstances might your flaw impede you? Why does your character have this flaw? What does this flaw say about your character's personality? You should try to come up with flaws that tell a story, provide insight into the character, and most importantly, indicate when you can invoke the flaw for Reputation Points.


\paragraph{Examples of well-written Flaws}

\begin{multicols}{2}

\begin{tcolorbox}
Flaw: Acts on impulse in social situations.
\tcblower
Weakness: My character didn't learn many social graces growing up on a farm, and often speaks his mind when he shouldn't. 
\end{tcolorbox}

\begin{tcolorbox}
Flaw: Has to drink after firing a gun or rifle.
\tcblower
Weakness: My character became an alcoholic while trying to forget the death and horror she saw in the war.
\end{tcolorbox}

\end{multicols}


\paragraph{Examples of poorly-written Flaws (Don't write Flaws like this!)}

\begin{multicols}{2}

\begin{tcolorbox}
Flaw: Impulsive
\tcblower
Weakness: My character insults people because he feels like it.
\end{tcolorbox}

\begin{tcolorbox}
Flaw: Alcoholic
\tcblower
Weakness: My character drinks too much.
\end{tcolorbox}

\end{multicols}

\section{Starting Gear and Miscellany}\label{Starting Gear}
Each character starts with a \enquote{kit} filled with normal quality items and some personal effects. These can be written in the \enquote{Gear} section of your character sheet.

\subsection{Your Kit}

Characters begin the game with a \enquote{kit} that represents what the characters bring with them to the table, and is usually based on who your character is and the tools of their trade. This kit represents the items you need to perform your Skills. You should consider each one of your Archetype skills and decide what you need to effectively perform each of those Skills to determine what's in your kit. \textbf{Your kit does not give you bonuses when using those skills,} but losing elements of your kit, or letting them be damaged or broken can result in penalties. 

\textbf{Everyone begins the game with a gun and a horse.} In addition, Working Folk begin with a melee weapon that reflects the type of work they have done (See Gear, pg. \pageref{Gear}). Your starting kit must be verified by the GM prior to starting the first session.

\textit{Example: Doctor Wieland begins the game with a Derringer, a Quarter Horse named Wanderer, and a Doctor's Kit. His Doctor's Kit contains everything that he needs to treat injuries on the fly.}

\subsection{Your Rhetorical Strategies}

Characters begin the game with one Rhetorical Strategy from each of the three categories: Logicality, Empathy, and Hostility (See Rhetorical Strategies, pg. \pageref{Rhetorical Strategies}). If a character has a rating of 3 or higher in any of the skills associated with rhetorical strategy categories (Education, Empathy, or Intimidate), they may choose an additional Rhetorical Strategy in the appropriate category.  Rhetorical Strategies are used for Combat, and inflict Strain upon a character.

\subsection{Other Character Details}

\paragraph{Personal Effects}

These are small items that do not affect your skills, but help you create a more detailed description of your character, such as a pipe, locket, or unique articles of clothing.

\textit{Example: Doctor Wieland begins the game with a set of doctor's clothes, photo of his father, a copy of his father's death certificate, and  warm scarf.}

\paragraph{Language and Literacy}

Characters are only literate if they have a rating of at least one in the education skill. 

\paragraph{Optional Rule} Poise Based Languages: Characters begin the game able to speak a number of languages equal to half their Poise Rating (minimum of 1 language) and are literate in a number of languages equal to half their Education Rating (both are rounded up). 

\paragraph{Starting Wealth}

Characters begin the game with \$2d6 unless they have the \enquote{A Life of Privilege} (\$2d6+70) or a \enquote{A Life of Struggle} (\$d6) Background.