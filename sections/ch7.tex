\documentclass[../main.tex]{subfiles}

%----------------------------------------------------------------------------------------
%	COMBAT
%----------------------------------------------------------------------------------------


\chapter{Combat}\label{Combat}

\lettrine[lines=2]{C}{ombat} in American Sunset is designed to be fast, dramatic, and decisive. Whenever combat takes place, the stakes are life and death - and the system is designed to exemplify this. Combat is distinguished in two ways: social combat and physical combat. Social combat takes the form of organized debates, heated interpersonal arguments, and tense negotiations, which may turn lethal. Physical combat takes place through exchanges of bullets, fists, and the occasional blades, arrows or improvised weapons. Both can coexist in the same combat, providing a multi-layered and dramatic combat experience.

\section{Structured Time}
All Combat takes place in \textbf{Structured Time} - a system of organizing actions into a series of turns and rounds. Whenever there is a situation where it is important to know the specific order of character interactions, the GM may institute Structured Time. During Structured Time, the order of players' actions are determined and play proceeds in the given order, in which each player takes a turn (see below for details on what comprises a turn). Once every character has taken a turn, the current round is over and a new round begins.

In American Sunset, there is both Physical and Social Combat. Because the game uses the same system to handle both, players are encouraged to behave dynamically. Characters use Rhetorical Strategies to inflict Strain and Weapons to inflict Wounds. Allowing for a variety of responses to a single challenge. Whenever the term \enquote{Combat} is used, it is synonymous with structured time. Structurally speaking, there is no \enquote{Physical} or \enquote{Social} Combat. All combat is resolved using the same system on the same Initiative count.

\subsection{Scenes}
In any combat or other structured time scenario, the environment is broken up into scenes. Scenes are designed to allow players to quickly and easily understand their position during combat, even in complicated or chaotic situations. Each scene is relatively small, easily described in a few words or a couple of short sentences, and relatively uniform in terrain and obstacles. Players are able to freely move inside a single scene, but moving from one scene to another involves suffering a -2 penalty to their attack.  (See Movement, pg. \pageref{Movement})

\textit{Examples of scenes: \enquote{In the corner, by the piano}; \enquote{In the middle of the street}; \enquote{Along the edge of the ridge}; \enquote{In the master bedroom}; \enquote{On the balcony}; \enquote{Next to the large group of cacti}; \enquote{In the shadow of the hill}; \enquote{Inside the caboose}; etc.}

\subsection{Range}\label{Range}

All weapons in the game have a range associated with them that determines their effectiveness over varying distances. These ranges do not have hard rules associated with them. instead, we've listed here three possible descriptions for each range bracket: the number of scenes composing the range bracket; an intuitive, narrative context; and a distance in units of measurement. All three are equally viable, and may be more or less suited to different situations. Use whichever is best for you! GMs are encouraged to be flexible and creative with this system. 

\subsection{Range Brackets}
\begin{description}
\item[Short] Within the same scene. The distance at which you can comfortably have a conversation while speaking normally. Zero to fifteen feet.

\item[Medium] One to three scenes away. The distance across a large room or a bar, or a distance that would take several seconds to cross at a sprint. Fifteen to forty feet.

\item[Long] Four to five scenes away. The farthest distance you could yell and be understood. Forty to eighty feet.

\item[Very Long] More than five scenes away. Too far to make out the recognizable details of a person. Eighty to one hundred and fifty feet.

\item[Extreme] At extreme distances, individuals are barely discernible. Circumstances where this range bracket may be needed will be very rare. Examples of extreme range include the distance from the peak of a mountain to the base, or the first sighting of riders in the distance on the open plains. Realistically, you should not be initiating combat at this range with anything less than a cannon.

\end{description}

\hfill

Characters who make a ranged attack at a range bracket farther than their weapon's listed range suffer -2 to their roll for each range bracket beyond the listed range.

\hfill

\textit{Example: Jessica wants to shoot a fleeing bandit with her revolver. The bandit is at Long range from Jessica (5 scenes away) and her revolver has a listed range bracket of Medium. When shooting at the bandit, she suffers a -2 to her roll.}


\hfill

Characters may not attack more than 1 range bracket beyond the range bracket listed for their weapon.
A character may not Counter-Attack a target that is beyond the range bracket listed for their weapon - they may only Evade. It is simply too distant for the character to accurately return fire.

\subsection{Initiating Combat}
When Structured Time is announced, determine which characters are aware and unaware. While sometimes this may be obvious from a narrative standpoint, you may also wish to have defending characters perform an Awareness contest to see if they really are unaware. If there is an unaware character, treat the first round of combat as an Ambush Round.

\paragraph{Ambush Round}\label{Ambush Round}
Whenever a character would attack an unaware character, the first round of Structured Time is the Ambush Round, in which unaware characters cannot act. When determining Initiative for the Ambush Round, do not roll Initiative for unaware characters. Unaware characters are unable to act or react during this round and they may not Counter-Attack.
Once the Ambush round is over, roll Initiative for unaware characters. In the next round, unaware characters become aware and act normally in subsequent rounds.

\paragraph{Turn Order and Initiative}\label{Turn Order}
The order of play is dictated by an Initiative Contest, in which each character rolls 2d6 and adds their Awareness Rating to the total. The order of play proceeds from the character with the highest total to the character with the lowest total. 

\subsection{Turn Actions}\label{Turn Actions}
During a character's turn, they may do one of each:
\begin{itemize}
\item Initiate a Contest (See Initiating a Contest, pg. \pageref{Contests})
\item Move to an adjacent scene (-2 to any Shooting contests made)
	\begin{itemize}
		\item If a character moves to an adjacent scene during their turn, they suffer a -2 penalty to any Shooting contest they make until the start of their next turn. 
		\item A character may choose not to initiate a Contest during their turn in order to make a Fitness contest with a Target Number of 8. If they are successful, they may move an additional scene during their turn. If a character succeeds at the previous test, they make another Fitness Contest with a Target Number of 15 and if they succeed again, they may move another scene. (See Movement, pg. \pageref{Movement})
		\item For Movement and Mounted Combat, see Mounted Combat, pg. \pageref{Mounted Combat}.
	\end{itemize}
\item Speak a sentence
\end{itemize}

A character may also pass, and do nothing during their turn. Characters will usually pass if they are attempting to make an attack with other characters against a single target.

\paragraph{Using Weapons}
Characters use Weapons to deal Wounds to other characters. All weapons have an associated skill. When making an attack, roll 2d6 and add the associated skill rating. If a character is attempting to Evade (See Opposed Contests, pg. \pageref{Opposed Contests}) an attack with a Weapon, they add their Grit to the die result instead.

\subsection{Cover}\label{Cover}
If a character is attacked with a ranged attack while standing behind cover, the severity of any Wounds the defending character receives is reduced by the type of cover that they are behind.

\begin{tcolorbox}
\begin{tabular}{ m{2.5cm} | p{5cm} | p{5cm}}
\textbf{Type of Cover} & \textbf{Example} & \textbf{Effect} \\\hline


 Soft Cover	& Wood. Thin Metal. A particularly robust window. 	& Decrease degree of wound severity once.			 \\  \hline
	
	
 Hard Cover	& Reinforced wood. Stone. Layered sandbags. 		& Decrease degree of wound severity twice.			 \\  
	

\end{tabular}
\end{tcolorbox}

\paragraph{Ignoring Cover} When attacking with a Weapon, attacking characters who roll a doubles on any dice they roll for the attack (including dice rolled with Advantage and Disadvantage) ignore the defender's cover when resolving their attacks.

\section{Mounted Combat}\label{Mounted Combat}
Horses are an iconic component of any Western tale, and American Sunset is no different. While horses count as characters like any other animal or entity, they are unique in that other characters, player or otherwise, can make use of them during combat. In keeping with this, horses and other mounts have qualities much like weapons and rhetorical strategies, and some can even have a reputation associated with them.


\paragraph{Health}

All Mounts have a shared Wounds and Strain track (an "Injury Track." See Antagonists, pg. \pageref{Antagonists} for more info about injury tracks) that is used whenever they take either physical or mental injuries. Whenever a Mount would take a Strain or Wound of any degree, they take an injury of that degree. Rhetorical Strategies cannot be used against mounts in combat.

Additionally, once a Mount takes a severe Wound, they will suffer an additional light Wound whenever their rider takes a movement action. As with all other characters, once a mount has filled each of their possible wounds, they are dead.

\paragraph{Movement}\label{Movement}

Unladen Mounts, by default, can move up to three scenes per turn as their first movement action. The rider may take another movement action, provided that they succeed at a TN 10 Riding Contest and have not initiated a contest this turn. Note this may not be true of some mounts such as donkeys or ponies! Additionally, Mounts do not receive their own turn during combat, as they are taking commands from their rider. 

If a mount is killed while being ridden, the rider is thrown and takes a wound. The severity of their wound is determined by how many scenes the rider traveled through in their previous turn.

\begin{tcolorbox}
\begin{tabular}{ | p{5cm} | p{8cm} |}
\textbf{Distance} & \textbf{Wound Degree} \\\hline

 0 - 1 	& Light 				 \\  \hline

 2 		& Medium 				 \\  \hline
	
 3		& Severe 					 \\  
	
\end{tabular}
\end{tcolorbox}

\paragraph{Mounted Attacks}

Characters riding a mount have several options in combat, some of which rely on the mount and rider traveling a number of scenes before they make an attack. Riders who have traveled more than two scenes in a turn before attacking may gain Advantage against their target, or they may choose to Trample their target instead of using a weapon. When doing so, they treat their Mount itself as a weapon, and use its qualities for the attack.

If a character chooses to Trample, or their target is counterattacking with a weapon that has the Polearm quality, they do not gain Advantage.


\section{Rhetorical Strategies}\label{Rhetorical Strategies}

Social Combat is a series of negotiations, movements, and arguments intended to achieve a single purpose. This could be swaying a crowd to your cause, convincing someone to help you, or navigating the often byzantine and corrupt legal system. Social Combats are initiated by characters who want to affect a change in someone's behavior. The arenas are court rooms, assembly halls, and bars. 

\subsection{Using Rhetorical Strategies}

Rhetorical Strategies are used to demoralize, distract, and disarm your enemies, as well as revitalize your allies. During combat, a character may choose to use a Rhetorical Strategy instead of attacking with a weapon, forgoing their chance to inflict a Wound in order to inflict a point of Strain. In order to Evade a Rhetorical Strategy, a character uses their Poise skill instead of the Grit.

Each Rhetorical Strategy falls under one of three categories, Logical, Empathetic, and Hostile. Each of the three strategy categories are associated with a skill; Education for Logical Strategies, Empathy for Empathetic Strategies, and Intimidate for Hostile Strategies. When using a Rhetorical Strategy, add the appropriate skill rating to the dice result in order to determine the total.

Note that Strategies do not have range brackets, because they can only be used inside the same scene (or within earshot, as defined by the GM).


\subsection{Logical Strategies (Education)}
For all Logical strategies, if Doubles are rolled, one targeted opponent (even if you are targeting multiple) may not initiate a contest during their next turn. 


\begin{tcolorbox}
\begin{tabular}{  p{2cm} | p{5cm} | p{5cm} }
\textbf{Strategy} & \textbf{Qualities} & \textbf{Other Effects}		\\\hline
Blather* 		& Underhanded 							\\  
			& Unreliable (5)	  &  Whenever a character loses to someone using this rhetorical strategy, the loser cannot initiate a contest during their next turn	\\ \hline
Blunt 		& Shocking (1) 								\\  
			& Reliable (5)								\\ \hline
Concise		& Underhanded 							\\  
			& Accurate (1)								\\
			& Compact								\\ \hline
Demand		& Shocking (2)								\\
			& Inaccurate (1)							\\
\end{tabular}	
\end{tcolorbox}

\subsection{Empathetic Strategies (Empathy)}
For all Empathic strategies, if Doubles are rolled, you may erase a Medium Strain on a character within your scene.

\begin{tcolorbox}
\begin{tabular}{  p{2cm} | p{5cm} | p{5cm} }
\textbf{Strategy} & \textbf{Qualities} & \textbf{Other Effects}		\\\hline
Butter Up* 	& Unreliable (5)	  &   A number of opponents equal to your Poise Rating suffer Disadvantage when using Rhetorical Strategies against you until the end of your next turn.	\\ \hline
Sincere 		& Accurate (1) 								\\  
			& Underhanded								\\ \hline
Innuendo*		& Accurate (1) 	& Can select any number of characters in a scene to discretely receive a short verbal cue.					\\  \hline
Inspire*		& Inaccurate (1) & All of your allies within your scene can erase their Light Strain.					\\
\end{tabular}	
\end{tcolorbox}


\subsection{Hostile Strategies (Intimidate)}
For all Hostile strategies, if Doubles are rolled and the strategy does not have Shocking (X), it gains Shocking (1) in addition to its other qualities. If the strategy already has Shocking (X), it now has Shocking (X+1) 

\begin{tcolorbox}
\begin{tabular}{  p{2cm} | p{5cm} | p{5cm} }
\textbf{Strategy} & \textbf{Qualities} & \textbf{Other Effects}			\\\hline
Indifferent 		& Reliable (4) 								\\  
				& Accurate (1)	  							\\ \hline
Mock 			& Underhanded 							\\  
				& Shocking (1)								\\ \hline
				& Inaccurate (1)							\\
Threaten			& Shocking (1) 								\\  
				& Reliable (5)								\\ \hline
Denounce			& Shocking (2)								\\
				& Unreliable (5)								\\
\end{tabular}	
\end{tcolorbox}

*Rhetorical Strategies marked with an asterisk do not have a injury stake and do not deal Strain when they win a contest, even if they are Counter-Attacking.

\subsection{Optional Rule: Identity Politics and Social Combat}
If the GM wishes, they may highlight the racism, classism, or prejudices of a social situation by detailing privileged identities and marginalized identities particular to it. When using rhetorical strategies, characters whose identities align with a privileged identity may invoke their privilege as if it were a trait, and characters whose identities align with a marginalized identity may invoke their marginalization as if it were a flaw. Narrative explanations of how these elements are invoked is necessary.

Whenever this optional rule is used, the GM should be clear with their players about the identity politics encountered in each social situation. This can be used to reflect more than the standard prejudices of white, patriarchal American society, such as the intense suspicion of foreigners harbored by Native tribes. The West is populated by small, disconnected communities with their own fears and prejudices and this guideline is designed to represent that. 

\section{Movement}
During a character's turn, they may make a Movement action. A movement action is taken when a character wants to move to an adjacent scene as a part of their turn. Characters on horseback may make three movement actions in their turn.

If a character succeeds at a Fitness contest with a TN of 8, they may move an additional scene during their turn. If a character succeeds at the previous test, they make another Fitness Contest with a TN of 15 and if they succeed on this test, they may move another scene. If a character moves more than one scene in a turn, they may not Initiate a Contest. Likewise, if a character initiates a contest, they may not move more than one scene during their turn. 

If a character moves between scenes and makes an attack in the same turn, that attack suffers a -2 penalty unless the Weapon or Rhetorical Strategy has the Compact quality.
\paragraph{Optional Rule: More Realistic Horse Movement}

A horse may not move from scene A to scene B, and then move from scene B to scene A during the same turn.

\section{Multiple Participants in Combat}\label{Multiple Participants in Combat}
You may often find find yourself in a situation where there are more than two characters in combat. The following rules  describe exactly how to play out these scenarios.

\subsection{A Single Character Attacking Multiple Targets}
\begin{enumerate}
\item The active character chooses how many targets they will be initiating a contest against. 
\item Stakes proceed as normal. The defending group must choose as a group to raise the stakes and will each take the appropriate penalty to all of their rolls for doing so. 
\item Characters roll dice and calculate their total. In addition, the attacking character subtracts 1 from their roll for each character they are attacking beyond the first. 
\item The attacking character's total is compared to each individual defending character's total. Each defending character with a lower total suffers an appropriate injury and any other stakes. Then, the attacking character suffers an appropriate injury each defending character with a total higher than theirs.
\end{enumerate}
\textit{Example: Jessica James is shooting at 4 bandits who are making trouble in her town. Since she has little patience for bandits, she wants to set the stakes at a severe wound. The bandits choose not to raise the stakes, and all parties proceed to roll and add their skill bonuses. Jessica has a high Shooting skill, so her total is a respectable 14, despite the -4 to her roll. Three of the bandits are somewhat less skilled, and their totals are 8, 10, and 6. However, the lead bandit is somewhat more skilled, and his total is a 15. The three bandits who rolled low each take a severe wound from Jessica, but the lead bandit deals a severe wound to her and takes no damage.}

\subsection{Multiple Characters Attacking A Single Target}
The characters intending to attack together  (the \enquote{attacking group}) must wait through Initiative until the last character's turn in the Initiative order (See Structured Time: Combat, pg \pageref{Combat}).The attacking group then chooses to initiate a contest, as a group, against a single target.
\begin{enumerate}

\item Stakes proceed as normal. The attacking group must agree on the stakes being set, and the defending player may raise them as normal.
\item Players roll dice, and the defending player receives a -1 for each attacking character beyond the first.
\item The defending character's total is compared to each attacking character's total individually. Each attacking player with a lower total suffers an appropriate injury stake and any other stakes.. \item Each attacking character with a higher total than the defending character takes the appropriate injury stake and any other stakes.
\item In the case where one side, either the attacking group or the target, has won all the individual resolutions, that side counts as the winner for all other relevant stakes. In other cases, the GM must determine an appropriate division based on the circumstances.
\end{enumerate}

\textit{Example multiple combat: Some bandits have heard about Banjo Tom's formidable skills in combat and want to team up to get an edge. They all roll Initiative and while the lead bandit rolls higher than Banjo Tom, the other two roll worse. The lead bandit has to wait for his allies to be ready since he wants to make a group attack so Banjo Tom gets to attack first. When the other bandit's turn in initiative comes up they choose to make a group attack against Banjo Tom. They set the stakes at medium and Banjo Tom chooses not to raise them. The bandits roll dice and Banjo Tom rolls dice at -2 since there are three attackers. Banjo Tom rolls an 8, the lead bandit and one of his cohorts both roll 9's but the third bandit rolls a 5. Banjo Tom takes two medium wounds and the bandit who rolled low takes one as well. The bandit leader and his other friend take no damage.}

This system does not allow for multiple attackers against multiple defenders because that's what normal combat already represents. A given combat scenario is broken up into individual 1v1, 1vX, and Xv1 combats.

\paragraph{Dual-Wielding} If a character is firing a pair of the same pistols, resolve the attack as if only one pistol is firing. That pistol gains the following qualities: 
\begin{itemize}
\item Inaccurate (1)
\item Brutal(1)
\end{itemize}
If a weapon with the Accurate quality gains the Inaccurate quality, the number (X) listed beside the Accurate quality is reduced by 1. If the weapon already has the Inaccurate or Brutal quality, the listed number (X) is increased by 1.

For more information on weapon qualities, see page {Qualties}.

\section{Wounds and Strain}\label{Wounds and Strain}

The world of American Sunset is a perilous one, fraught with dangers and risks. Be it through combat, accidents, or even overwhelming terror, injury is inevitable. Characters have two different ways of tracking their health: Wounds, which measure physical injuries, and Strain, which measures mental injury.
	
Both Wounds and Strain are broken into three categories, Light, Medium, and Severe, measuring how badly the character has been injured. Character sheets have boxes arranged in order of severity and are marked whenever a character suffers an injury.
	
Characters with a higher Grit can withstand more Wounds before they are incapacitated, and similarly, characters with a higher Poise can withstand more Strain before they are incapacitated.

\begin{multicols}{2}
\begin{tcolorbox}
\begin{tabular}{  p{2.5cm} | p{3cm} }
\textbf{Grit Rating} & \textbf{Max Wounds} 						\\\hline
0		 		& 3 Light	 								\\  
		 		& 2 Medium 								\\  
				& 1 Severe	  							\\ \hline
1-3	 			& 4 Light 									\\  
				& 2 Medium								\\ 
				& 1 Severe								\\ \hline
4-7				& 4 Light 									\\  
				& 3 Medium 								\\  
				& 1 Severe								\\ \hline
8-10				& 4 Light									\\
				& 3 Medium								\\
				& 2 Severe								\\ 
\end{tabular}	
\end{tcolorbox}


\begin{tcolorbox}
\begin{tabular}{ p{2.5cm} | p{3cm} }
\textbf{Poise Rating} & \textbf{Max Strain} 						\\\hline
0		 		& 3 Light	 								\\  
		 		& 2 Medium 								\\  
				& 1 Severe	  							\\ \hline
1-3	 			& 4 Light 									\\  
				& 2 Medium								\\ 
				& 1 Severe								\\ \hline
4-7				& 4 Light 									\\  
				& 3 Medium 								\\  
				& 1 Severe								\\ \hline
8-10				& 4 Light									\\
				& 3 Medium								\\
				& 2 Severe								\\ 
\end{tabular}	
\end{tcolorbox}


\end{multicols}

Characters who suffer injuries take penalties to future contests depending on the severity of the injury.

\begin{tcolorbox}
\begin{tabular}{ p{2.5cm} | p{9cm} }
\textbf{Injury degree} & \textbf{Consequence} 											\\ \hline
Light					& No Penalty	  											\\ \hline
Medium				& All contests are rolled at -1 for each injury of this type				\\ \hline
Severe				& All contests are rolled at Disadvantage							\\ 
\end{tabular}	
\end{tcolorbox}

To offset these consequences, characters must treat their injuries. This is accomplished in one of two ways, treating Wounds, and treating Strain.

\subsection{Treating Wounds}\label{Treating Wounds}
To treat Wounds, a character must perform a Medical Contest. The Target Number for the medical test is determined by the kind of Wound a character is attempting to treat.

After treatment, Light Wounds are removed entirely. However, Medium and Severe Wounds are not. Instead they are Bandaged, represented by the smaller box next to Medium and Severe wound boxes on character sheets. Wounds that have been Bandaged are not gone, but no longer inflict any lasting effects on the character who has suffered them. After a number of days based on the severity of the wound minus your character's Fitness rating, the wound is healed.

In Structured Time, treating a Wound takes a number of rounds equal to the Target Number of the Wound minus the treating character's Medical Skill (minimum 1 round). During this time, they may not move or perform any other contests, though they can still speak as normal.

If you are treating your own Medium or Severe wounds, add 2 to the difficulty. 

\begin{tcolorbox}
\begin{tabular}{ p{2cm} | p{1cm}| p{3.5cm} | p{2.5cm} | p{2.7cm} }
\textbf{Severity} & \textbf{TN} & \textbf{Example} & \textbf{On Treatment} & \textbf{Time to Heal} 										\\ \hline
Light 	&	5 	&	 Just a scratch. Sprained joint or pulled muscle. & Removed & Instant						\\  \hline
Medium	&	10	&	 Bruised or fractured bone. A profusely bleeding cut. & Bandaged & 10 minus Fitness days		\\  \hline
Severe 	&	15	&	 Badly broken or shattered bone. Bullet to the chest. & Bandaged & 15 minus Fitness days		\\  
\end{tabular}	
\end{tcolorbox}

\subsection{Treating Strain}
To treat Strain, a character must perform an Empathy Contest. The Target Number for the Empathy Contest is dictated the degree of Strain that the character is attempting to treat. Before making the Empathy Contest, the character must announce the level of Strain they are trying to treat. 

In Structured Time, treating Strain takes a successful Empathy Contest for each point of Strain that the character is trying to remove. Light and Medium Strain are gone after treatment. Severe Strain is not gone after treatment, but it stops providing Disadvantage to rolls once it is treated. Severe Strain doesn't go away immediately after treatment although the injured player will stop taking Disadvantage once it has been treated. After treatment it will still take time for the Severe Strain to be completely gone. After 5 days you may make another Empathy contest to remove the wound entirely.

\begin{tcolorbox}
\begin{tabular}{ p{2cm} | p{1cm}| p{3.5cm} | p{2.5cm} | p{2.7cm} }
\textbf{Severity} & \textbf{TN} & \textbf{Example} & \textbf{On Treatment} & \textbf{Time to Heal} 										\\ \hline
Light 	&	5 	&	 Bruised ego. Temporary social awkwardness. & Removed & Instant						\\  \hline
Medium	&	10	&	 Shocked or afraid to the point of taking pause. & Bandaged & Instant		\\  \hline
Severe 	&	15	&	 Shellshock. A moderately debilitating panic attack. & Bandaged & 15 days. (Or 5 days, if treated)		\\  
\end{tabular}	
\end{tcolorbox}


\subsection{Resolving Injuries: Death and Surrender}\label{Death and Surrender}
As gameplay progresses, a character will probably suffer an injury when they have already taken their maximum injuries of that severity. In this case, they instead take an injury one degree more severe than the injury they would otherwise take.

\textit{Example: Adi has already taken 4 light wounds, and is dealt another because she trips over a loose rock. Because she does not have a fifth space to fill, she instead takes a medium Wound. 
If a character is dealt a Strain and all of their Strain have already been allocated, they have no choice but to surrender and  cease to be an active participant in combat. For narrative purposes, they will reluctantly accede to their opponent's demands.}

If a character is dealt a Wound and all of their wounds have already been allocated, they are dead.

In the case of non-Reputable characters, this is the end of the line for them. However, Reputable characters are made of tougher stuff - If they're treated soon, they may have a chance to survive. This isn't a situation to take lightly, however! Characters who've taken their maximum number of wounds have experienced a seriously traumatic event. Outside of extraordinary circumstances, you should strongly consider retiring the character. If the character is not retired, the GM may decide to give that character an additional flaw to reflect the trauma.

\section{Qualities}\label{Qualities}
Weapons, Rhetorical Strategies, and even some other game objects such as horses are imbued with \textbf{Qualities}. Qualities distinguish different types of combat devices, add either add additional effects or make a device more difficult to use.

For the purposes of Qualities, a Counter-Attacker is also an attacker, unless otherwise stated. If, for whatever reason, a weapon or rhetorical strategy would gain a quality that it already has, increase the X value of the quality by 1. If a weapon or rhetorical strategy does not have an X value to increase, there are no changes to the weapon's qualities. 

\begin{multicols}{2}

\begin{tcolorbox}[title= Accurate(X)]
Add X to the your total when making an attack with this weapon or rhetorical strategy. 
\end{tcolorbox}

\begin{tcolorbox}[title=Brutal (X)]
Whenever a character attacking with this weapon or rhetorical strategy inflicts injury, they deal an additional number of Light Wounds equal to X. 
\end{tcolorbox}
	
\begin{tcolorbox}[title = Concealable]
If a character with this weapon is searched, the character being searched receives a +2 to their attempts to conceal this weapon. 
\end{tcolorbox}

\begin{tcolorbox}[title = Compact/ Mobile]
Do not take a -2 penalty when attacking with this weapon if the character has taken a single movement action since the beginning of their last turn. If the character has taken more than one movement action, they still cannot make an attack.
\end{tcolorbox}

\begin{tcolorbox}[title=Inaccurate (X)]
Subtract X from your total when making an attack with this weapon or rhetorical strategy. 
\end{tcolorbox}

\begin{tcolorbox}[title=Non-Lethal (X)]
You may not deal severe Wounds or Strain with this weapon or rhetorical strategy unless your relevant skill rating is equal to or higher than X.
\end{tcolorbox}

\begin{tcolorbox}[title=Polearm]
A weapon with this quality can used to deprive a mounted attacker (who has taken two or more movement actions) of their Advantage. Mounted attackers may not make an attack with a Mount weapon against a character wielding a weapon with this quality.
\end{tcolorbox}

\begin{tcolorbox}[title=Quick Draw]
As long as this weapon is within reach, the character never counts as unaware for the purposes of an Ambush Round. If an unaware character has a weapon with this quality at hand, they count as aware for the Ambush Round, provided that they respond to any contests this turn with a Counter-Attack by using a weapon with the Quick Draw quality.
\end{tcolorbox}

\begin{tcolorbox}[title=Reliable (X)]
When attacking with this weapon or rhetorical strategy, if the character's dice roll lower than (X), they automatically count the result as (X).  
\end{tcolorbox}

\begin{tcolorbox}[title=Shocking (X)]
 Whenever a character attacking with this weapon or rhetorical strategy inflicts injury, they deal an additional number of Light Strain equal to X. 
\end{tcolorbox}

\begin{tcolorbox}[title=Slow]
A character may not Counter-Attack with this weapon or rhetorical strategy. 
\end{tcolorbox}

\begin{tcolorbox}[title=Spray]
When attacking with this weapon or rhetorical strategy, instead of nominating a single target, nominate every character within a single scene. Resolve attacks against all characters within this scene, friend and foe, using the Attacking Multiple Targets rules on p. \pageref{Multiple Participants in Combat}. 
\end{tcolorbox}

\begin{tcolorbox}[title=Unwieldy]
An attacking character using a weapon with this quality takes a -2 penalty when attacking a target within the same scene as the attacker. 
\end{tcolorbox}

\begin{tcolorbox}[title=Unreliable (X)]
When making an attack with a weapon or rhetorical strategy with this quality, if the attacker rolls underneath (X) on their dice roll, they automatically lose the contest. Whenever a character loses a contest in the manner described above, the weapon is considered broken until an appropriate Handiness contest is passed.  
\end{tcolorbox}

\begin{tcolorbox}[title=Underhanded]
The first time that you make an attack with an Underhanded weapon or rhetorical strategy in combat, your opponent must succeed at a Perception contest with a TN of 10 or be considered unaware for the purposes of the attack (i.e. they may not Counter-Attack in the Opposed Contest and must Evade). 
\end{tcolorbox}

\end{multicols}
