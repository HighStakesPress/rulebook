%----------------------------------------------------------------------------------------
%	PART
%----------------------------------------------------------------------------------------
\documentclass[../main.tex]{subfiles}

\part{Setting}

%----------------------------------------------------------------------------------------
%	CHAPTER 1
%----------------------------------------------------------------------------------------


\chapterimage{Great_Plains_1x2.jpg} % Chapter heading image

\chapter{History, Factions, and Characters.}

\lettrine[lines=2]{W}{elcome} to the world of American Sunset! This chapter tells the story of American Sunset's history, and lays down the important characters and organizations. They interact in subtle and overt ways, and create an interesting and dynamic social network for your characters to live and move in. The future actions of the characters in this chapter are not strictly set forth and they are as much tools to make this world interesting as the rules in Chapter 3. 
\section{City of Silver Springs}

Silver Springs is the capitol city of the state Jefferson, and the largest city for more than a hundred miles in any direction. Many of the efforts to expand into the few remaining unmapped regions of the frontier are based here, as are a hundred and one industries trying to compete for business in the quickly growing city.

\begin{multicols}{2}

\paragraph{History}
Before settlers arrived in Silver Springs, there were the Native Americans who would eventually compose the Civilized Five. They lived on the Jefferson Stripe, the massive river that runs through Jefferson territory, and upon whose banks, and upon whose banks  the  the southwestern edge of Silver Springs lies lies. Early settlers would tell stories of the rejuvenating power of the water and air, the golden glints in the river's ripples, the magic springs that gave worthy visions of the future, and the mysterious the mysterious raven migrations. Now these stories are scarce. As reality appeared to settle in, the stories disappeared and the hardships began to mount. As reality appeared to settle in, the stories disappeared and the hardships began to mount. A series of terrible floods have afflicted Silver Springs, winters seem to get harsher as the secrets of the mountains are plundered, and drought and fires occasionally rip through downtown, burning nearly all the businesses to the ground.

Like most other Jefferson Territory towns, Silver Springs emerged in the early 1850s during the silver rush. It was a small town in a fortunate place, surrounded on all sides by gold and silver rich hills. While the mineral value and renowned hot springs of the region were major influences on settlement, tales of the mystical and far-fetched also attracted a unique circle of settlers.  And it was its settlers -- not the Jefferson Stripe, the Plains that led up to it, or the mineral rich hills -- that made it what it is today. Perhaps the most extraordinary thing about these individuals, whose reputation precedes them, is that they are still our neighbors.

The first of these visionaries to settle this land of peaks and prairies, amongst the disowned braves, the hopeful miners, and the plucky wildmen of the frontier was Elias Reach. If you were to say that Silver Springs exists because of Elias Reach, you might just be right. Reach was once a surveyor, imposing the law of cartography on Oregon and California, before he traveled to the relatively untamed Jefferson State and likewise set it to paper. There, Reach set up shop as the editor of the Silver Springs Weekly, and to this day still prints the news exactly as he sees it. The Silver Springs we see today is a reflection of the city he predicted would one day come about. More than just a charismatic printer, Reach's influence gave rise to both the vigilante committees and city marshals, as well as the town jail, a building that he owns. Even the administration of capital punishment has been known to happen on his authority, although he styles it as \enquote{the common will of the people.} Reach is appreciative of diversity and freedom in his town, but threatening him or Silver Springs is a quick way to end up on the wrong side of his shotgun. 

By the time Reach established his paper, the gold rush had begun to wane, and the magical time, short though it was, in which hope and a pick was enough to make a living, had ended. Silver and gold was still produced in abundance from the ranges and caves around the town, but only at the hands of dedicated miners. Farmers began to settle and tame the land and record the seasons. 

Prospectors, who had done very well for themselves, began to influence the development of the town. Richard Crabb, a successful prospector-turned-owner of a real estate company out of Turage, managed to successfully buy out (and claim) some very choice land between the town of Silver Springs, the Highlands encampment , which was hardly a town at all, and Aproi, a rival town. Reach and Crabb began negotiations to combine these communities into one larger municipality, which would become the capital of Jefferson State. A central issue: where would mainstreet be located, in Aproi or Silver Springs? 

When the mayor of Silver Springs, Maximillian P. Woolcott (the man who is still mayor to this day) managed to convince the coach lines to run through Silver Springs rather than Aproi, tensions grew. The people in  smaller camps, like the Highlands, were hardly unified enough to protest, but the peoples of Aproi and Silver Springs would not back down. With the new coach lines, the Aproi supporters became more disheartened. Additionally, Herb Wyatt was able to convince the founders of Summit Bank, successful prospector Dale Gibbs and mining outfitter Benjamin-Francis Palmer, that it would be safer for them  to set up shop in the town with the Vigilante Committee in Silver Springs. The end of the struggle was near. When the now forgotten Aproi newspaper print shop  mysteriously burned down and Reach graciously extended the Silver Springs Weekly coverage to Aproi, Silver Springs took on its current form, becoming the capital. 

The town slowly attracted businesses and settlers, who were lured by the cheap property rates, the idea of the West, and an entrepreneurial spirit. Saloons and brewing companies were very lucrative and many other businesses from blacksmith shops to bakeries began to open up. 

The last, but certainly not least, of the truly successful Silver Springs prospectors was Herbert Wyatt. After procuring silver and copper from mines only he knows the location of, Wyatt reinvented himself as a gentleman gambler. Wyatt invested early and heavily into the saloon and brewing business immediately before the Civil war and made a second fortune. He is a self-made man, but  how he got his code of honor is anyone's guess. He's also interestingly a scholar and, in an attempt to create a scholarly community, sent letters to fellow academics tired of the ivory tower and armchair thinkers of the civilized world, inviting them to move West. With Reach's support and advertising, he successfully created a scholarly community, albeit a small one, in the town. Our current Gov. Campbell is still a member of Wyatt's Readers, as is the esteemed local lawyer and town elite Lawrence Myers. 

In the period  just before the war, Silver Springs was characterized  by slow growth and development, as the locals fought amongst each other and against nature, in the struggle to become a legitimate city in the West. After the war, however, it prospered. The establishment of Sprigg's JDSF (Jefferson-Deseret San Francisco) Railroad and the connection of Silver Springs to the American Continental rail line meant that cargo, goods, military transports, and ordinary people could come into and out of the region much more frequently and efficiently. 

Since the population boom, more and more of the State has been charted for railroad development, and organized mining continues to excavate fortunes out of the mountains. The lifestyle of Native peoples continues to be challenged, and their presence in the city is minimal and aggressively contested.

Safety is not a given in the town of Silver Springs.  Because mining is a key industry, the town of 20,001 residents (by census) is full of thugs and vagrants, holing up in the town waiting for their next big break. They spend their money on drinking, gambling and prostitutes, leaving what little they have left to go towards room and board. They are why saloons are successful in the West. Ever since the gold rush there have been thieves and murderers in a town whose people are so largely focussed on their own fortunes that no one really looks out for each other. Reach and others managed to finally get a jail, six police and a marshall, but even the monthly hangings only slow crime, not prevent it. With the influx of displaced soldiers after the war the situation has only gotten worse. 

Silver Springs has experienced  concentric rings of development. The center of town is now the business district where the movers and shakers of Silver Springs work. The newspaper headquarters are also there, along with the banks, the town hall and any other lucrative or long standing business venture. Due to a fire that swept through the city center during the war, everything there is now built of brick. Just outside the ring of brick buildings there are the taverns, newer businesses, inns, and some houses. This is the most varied district. It includes many different kinds of shops, craftsmen workshops, squares to gather in, and taverns to drink at. It is noisy and constantly in motion nearly all times of the day. Farther out are the newest houses; these are usually little more than shacks with hard packed dirt floors, but they are large in number. The people living in them are often displaced veterans, uncared for migrants low on luck and money, people unliked or unwanted by the community, like Chinese or black people, and other unfortunate people perceived as being different. Alice Gage's brothels are scattered throughout the rings. She has more authority than anyone would like to admit, so she runs her business the way she pleases.

The tallest building in town is the clock tower next to the train station on the North side of town. It is just a little bit removed, but it can be seen from almost all places in the town. Trains run all hours of the day. Just next to the station is a military fort, although it is poorly manned. 

It is a vibrant town, the hub of activity taking place in the Jefferson State. All the money comes into and out of Silver Springs, all the supplies and nearly all the people. 
The Major Characters of Silver Springs run the gamut from the politically inclined to lucky prospectors to folks looking to escape their destinies.

\end{multicols}

\subsection{Silver Springs Politicians}

As a hub for growing enterprise, many who hope to rise in the ranks of the US government have begun to descend on the city, searching for opportunity and fame.

\begin{multicols}{2}


\paragraph{Mayor Woolcott}
              Born into a notoriously political family, Woolcott is no stranger to controversy. Maximilian P. Woolcott's father was the governor of an eastern American state, which remained indecisive during the civil war, and relations with the new union were frigid afterwards. While the state's population were decidedly against slavery, Woolcott's father feared the devastation caused by war. He placed his dreams of success in his son and spirited him through his education. Woolcott was appointed Mayor of Silver Springs shortly after the death of his father; a tragedy which he never entirely recovered from. He became shrewd and serious, preferring isolation to company. 
His assistants describe his work ethic as impeccable, and his efforts made Silver Springs the hotspot of the Central West. He is getting on in years but remains a  sprightly and fearsome swordsman. As a celebrity fencer, he is known throughout the states for his legendary composure and remarkable precision.
Mayor Woolcott is ultimately concerned with consolidating his power over the fledgling state. This involves subsidizing infrastructure projects, something he and President Wolfe have personally discussed. The majority of this money comes from G.M.Holt, who is trying to grow his power into a state monopoly.  Wolfe has personally tried to curb Holt's influence by providing personal financial support to Mayor Woolcott, but the mayor is only too keen to develop his land and accepts both Wolfe and Holt's contributions with generosity. He has been in office longer than anyone else in the state and is thus the locus of the state's developing power which, through Grant and Holt, he has begun to centralize on himself and his office. With his upcoming reelection, there is no telling how the office of Mayor may change. 

\paragraph{Gov. Campbell}
Gov. Campbell was Silver Springs' acclaimed doctor before running for Governor. Outwardly, Dr. Peter Sydney Campbell appears to be a very stern old man: tall and skinny with a carefully sculpted grey beard along his chin, a completely bald head, and spectacles above his hawk like nose. Even his eyebrows are somewhat threatening, jutting out in dangerously owlish levels; however, any of his patients will tell you he has impeccable bedside manner and a warm smile. Being the only good doctor in Silver Springs, he is known to almost everyone in town, and when he ran as a \enquote{well-read, unbiased, and dependable man,} there was very little opposition. 
His aspirations for Jefferson are high, and largely directed toward the public good. His first real act of authority as governor has been to build a hospital and install his old clinic staff as its head staff. St. Francis opened very recently and is one of the only large buildings located not in the center of town, but among the poorer outer circle. He still remains allied with the scholars and visionaries of Silver Springs, but his discussion of higher taxes is giving fire to Mayor Woolcott's campaign for governorship, and the contest between them is growing quite heated: those who want to develop Silver Springs into a better place for the people support Campbell versus those who want a more commerce-oriented future champion Woolcott.

\end{multicols}



\subsection{Silver Springs Establishments}

Business is booming in the city of Silver Springs, and these are the people and companies who are making sure it stays that way.

\begin{multicols}{2}


\paragraph{American Continental}
	The story of American Continental is really the story of Sylvester Van Wyke. Wyke was born in New York. His father, Ernest Van Wyke, was a shrewd businessman who owned a successful shipping yard. As Sylvester got older he saw the trains moving goods to and from the yard and realized that was the future. Years before the civil war he acquired a small railroad company and, thanks to the business acumen drilled into him by his father, his railroad became a booming success. At the height of his success, however, he was voted off the board when he refused to sell to the then successful Atlantic Railways. Van Wyke's response was to begin again, acquiring another small railroad, building it up the way he had his previous one. Then, when a corruption scandal struck Atlantic Railways, severely crippling the company's image, he made his move. He bought Atlantic Railways and merged it with his own, naming the new combined railroad American Continental. At this point Van Wyke had developed a reputation for being an unstoppable businessman. When the war ended and the Transcontinental Railroad was to be built, there was no question of who would be building it.
	
	It was around this time that Van Wyke met Adrian Steele. Anyone who's been with the two of them would compare it to an unstoppable force meeting an immovable object, although they'd be hard-pressed to tell you who had the upper hand. Van Wyke uses the U.S. military to guard his trains and construction yards, Steele uses Van Wyke's trains to transport his troops and resources to the frontier to further his crusade against the natives. Despite their partnership they are constantly at each other's throats, neither of them could have gotten where they were without constantly exploiting the weaknesses of everyone they meet, a lesser man in either of their shoes wouldn't last a day.

\paragraph{G. M. Holt}
Gabriel Marshall Holt is the proud owner of G M Holt Bank, the most aggressively expanding bank on the frontier. Holt came to Silver Springs from Philadelphia just after the war to manage his empire's Westward expansion more personally. Holt is shaped like egg, with a round face and round hands, and what little neck he has to his name is hidden in the tuxedos he perpetually wears. Complete with a top hat and mustache, this enterprising walrus of a man is a figure of notice in Silver Springs. Despite his outward image, he has a very shrewd mind and very quick fingers, kept dexterous by trimming his rose garden that is his pride and joy. No small rose garden either, as he has more of a \enquote{rose farm,} and it takes up 3 acres of elevated beds on a property East of Silver Springs. The garden is his pride and joy and the delicate side it evokes in him is present in his love of opera and what he considers enquote{High Arts.} Delicacy should not be confused here with sensitivity or empathy; he is very frugal and charity is not in his vocabulary. 

His alliances with other investors are not friendships, and the train operators that bring his money into and out of Silver Springs seem to always be covered in cold sweat, yet their turnover rate is quite low. The military guards that come with the trains seem less afraid, but they still follow his directions to a tee. His motivations have always been financial but there are rumblings that his interests may be spreading to politics?

Mrs. Holt shares her husband's love of the \enquote{High Arts,} although she appreciates roses slightly less. She is a tall skinny woman with silvery blonde hair and a very sharp green-eyed glare, who can always be found in the latest fashions of American aristocracy, not that there are very many to notice in Silver Springs. Now that all of their children are out of the house, managing other branches of the bank or married, she has found new ways to direct her time. 

Leslie Harriet Holt considers herself a leader in the social reformation of Silver Springs, and while she stepped down from her position of head of the Vigilante committee some years back, her tea group is constantly discussing ways to bring culture to the somewhat disgusting denizens of the city. Getting on her bad side means getting on the wrong side of many of the rich women in town, which is not necessarily equivalent to the most influential women in town. Nonetheless they do have pull when it comes to the various popularity contests in town, and people on their bad side quickly find disparaging poems, songs, articles, and jokes at their expensive circulating through town with alarming speed. Any discussion of projects related to building opera houses, dancehalls, or music venues can be attributed to Mrs. Holt, although her only recent successful endeavor has been to create a horse-racing tack. She can be found there at every race appropriately cheering on the G M Holt stallions.

This strange mixture of interests, talents and wealth has endowed the Holt family with the ability to financially and socially outcompete most newcomers in their enterprises, especially in banking. Opposing them is to incur both a financial struggle and a social one, as investors have been trained to shy away from their competitors so as to not look bad if sudden character assassinations occur. With this looming threat on the board it's not surprising that G.M.Holt is the banking monolith that it is.

\paragraph{Meyers and Son}
Silver Springs has its fair share of crime. Business should be booming for local lawyer and prosecutor, Lawrence Meyers, but he's an elderly man looking to retire. Born and raised in Silver Springs, Meyers is the last in a long line of solicitors working at Meyers and Son. Tall, but slightly stooping, Meyers' craggy face hides a sharp tongue and a curmudgeonly nature. Always handsomely dressed, someone is paying him more than his regular salary. Rumor has it that he works for Mr. Van Wyke.  
Meanwhile, his junior partner, Luis Dantes, is poised to take over the law firm. Dantes has worked diligently, not only to earn Meyers' respect, but also the mayor's. Dantes is ambitious, and Meyers knows it. Perhaps, that is why Meyers hasn't retired.

\paragraph{Silver Springs Weekly} 
The Silver Springs Weekly is a typical Western newspaper. It focuses on politics, local events, social and cultural news and, of course, advertisements. Founded by Elias Reach, it is run by Bob Franks: editor, journalist, and critic. 

No one knows much about Mr. Franks except that he is a difficult and dangerous man. Short and slightly overweight, he doesn't cut an imposing figure, but everyone is afraid of him. No one in town knows whose side he is on. He is a middle-aged muckraker in both senses of the term: he is a man who investigates corruption and criminal activities, but he's also a scandalmonger. He lives in the shadows, smoking his pipe, always listening to the whispers on the wind.

Rumor has it that he is a major investor in American Continental (AC), but it may just be that: a rumor. The Silver Springs Weekly both praises and censures the actions of AC's owner, Sylvester Ernest Van Wyke. Mr. Van Wyke clearly avoids Mr. Franks, and so should you, unless you want to become front-page news. 

\paragraph{Summit Bank}
Summit Bank is a local bank run by Dale Gibbs and Benjamin-Francis Palmer. These two men, a wildly successful prospector and a local mining outfitter, respectively, had humble beginnings. Gibbs came to Palmer after several of his fellow miners had been tricked into selling their findings for pennies on the dollar. Palmer was sympathetic and began posting current gold prices outside his store. Soon Gibbs suggested he just exchange the currency himself. Together that's just what they did, the trust of the local miners worked to their benefit, and they quickly became a serious bank.  

Between Palmer's finances, and Gibbs' newly discovered intuition for playing the gold market, the two of them have laid their claim on the banking fortunes of Silver Springs and are now trying to outperform powerful East-coast bankers like Holt. What they lack in resources, however, they make up for in accessibility and hospitality. Residents of Silver Springs universally love Gibbs and Palmer, who they see as the \enquote{average man's} banker in contrast to Holt's financial monolith.  Yet, even if Summit Bank is beloved by the people of Silver Springs, they have not escaped Holt's notice. People are expecting Holt to react quickly and decisively in eliminating Summit Bank, but strangely nothing has happened as of yet. Perhaps they are discussing a merger?

\paragraph{Alice Gage}
The world's oldest profession was one of the first to follow prospectors and settlers into the West, and Alice Gage is one of the most successful madames in the history of the West. It's a fair bet that every prostitute in Silver Springs is an employee of Gage's, and you'd be hard pressed to find any in the state of Jefferson who has not worked for her at some time.
A buxom and intense woman, interactions with Gage are frequently one sided, and those who meet her often mistake her for being much taller than she is.

Gage's personal history is a  mix of mystery and tall tales, misdirection that she gladly encourages. Rumors that she was descended from European royalty, an exiled Enochite, or the child of the unholy union between an excommunicated Enochite preacher and a demon are all true, at least according to her. In reality, she was a Chicago prostitute with a good ear for business opportunities and the will to claw her way into a position of power.

As the head of the largest brothel in Silver Springs, and owner of all the rest, Gage is one of the most well-informed, and well-moneyed people in the city.

\end{multicols}

\subsection{The Silver Springs Bounties}

The message boards of Silver Springs are always plastered with dozens of fading posters looking for work, but these four cases have managed to avoid being removed for some time. Too much time, if you ask Sheriff Hammond.

\begin{multicols}{2}

\paragraph{The Witch of Vander's Gulch }
The story of the Witch of Vander's Gulch is shrouded in mystery. Legend has it that Vander's Gulch was the place where the Great Spirit began creating the world, and filled the Gulch with her failed creations once the world was formed. It is said that, although the gulch was filled with monsters and abominations, it was the most peaceful place in the world because a young woman had pacified its inhabitants. No one knows where she came from or what she did, but she made the meek from the monstrous, and her power was renowned. One night, when a full moon stole the night sky and the Trickster Fox was laughing, the monsters took up tendencies befitting their grotesque visages and laid waste to the gulch. The Trickster Fox approached the bereft woman, who wept for her dead friends, and the fox cursed her to always look like those who she tried to save. She became a hideous crone, whose appearance curses her visitors to repeat her tragedy. They fall victim to their own kindness, only to be betrayed by those they helped. Even her name is unmentionable amongst settlers- especially Enochites, who must bless the air to dispel her inauspicious name. The Witch's curse manifests in seemingly harmless ways, and many attribute the existence of the curse he bitterness of those who believe themselves cursed. It begins by refusing coins to beggars and ends with cynicism and paranoia; one day, those who need your help will leverage your empathy and take everything from you. It becomes an obsession for those who believe themselves cursed: that those that you help rob you of your altruism.

It is said that the Witch of Vander's Gulch still has the purest heart in the west, and has never intentionally cursed anyone. She has completely isolated herself within the gulch and refuses to take visitors. Yet, even in the gulch, the origin and wastebin of all creation, she cannot escape the doomed landscape. The irony is that she perpetuates the cruelty of the land, yet she is the kindest soul of all. Amongst the native tribes, young men are sent into the gulch to confront a paradox of reality - that those who commit their lives to saving others are often consumed by them. She remains there to this day, and serves as a warning for those who wish to help others, cursing them for their benevolence. The curse remains to be verified, but it is certainly believed.

\paragraph{The Perpetrator of the Gartner Homestead Slaughter}
	The Gartner family were some of the first homesteaders in the area. Self-sufficient on their farm, their absence from town went unnoticed until the entire family was found dead by a group of travelers seeking shelter from a storm. Upon investigation, Marshall Hammond determined they were killed with a knife, but the weapon was never found on the grounds. In the wake of the discovery, a volunteer group was sent to alert other nearby settlers and they discovered that a similar fate had befallen the more secluded Smith homestead. The Smiths had been killed months earlier, but due to their isolation, their deaths had gone unnoticed, without anyone's knowledge. The common theory is that the murderers were natives, but the Marshall believes it was something much more sinister.

\paragraph{The Strange Case of the  JDSF 4:10 from Omaha}
	The disappearance of the JDSF railway 4:10 from Omaha is infamous in Silver Springs as one of the greatest unsolved cases of all time. Scheduled to arrive in Silver Springs on a clear spring afternoon, the train, her cargo, and her passengers simply never arrived. Later arrivals reported seeing absolutely no signs of it along the tracks and none were ever found. The train manifest has been checked and rechecked, and nothing of unusual value was onboard, as most bank and government cargo is transported via American Continental. An investigation into the passenger list was made as thoroughly as could be afforded, but never turned up any results. The current working theory is that the train was stopped on the bridge over the Jefferson Stripe and derailed over the side onto a raft, where it was then floated down the river. Countless issues with this theory have been voiced, but without an alternative theory, none have gained traction. The passengers are presumed dead since none of them have resurfaced in the months following the event.

\paragraph{Maria Ortega}\label{Maria Ortega}
Maria Ortega is the leader of the female bandit gang that has been conservatively taking a piece out of everything going into Silver Springs. Ortega is a short plump Hispanic woman with a warm smile and a taste for expensive revolvers. She began her career of thievery as the organizer of the wives of a bandit group called Los Diablos Argentes. They would chat, sing, watch the children and prepare fiestas for the men when they returned, until they didn't return after an attempted bank robbery gone wrong. Discovering that their husbands had been killed or arrested, she organized the women into a raiding group that managed to successfully break the surviving gang members out of jail. Discovering a taste for excitement, and with the prospects of a new empowering lifestyle, the women took up their husbands' mantle, and through a combination of careful planning and caution, have been a lot more successful in their raiding enterprises. Maria's passion is dancing, and Los Diablos send out discreet invitations for fiestas they hold outdoors every full moon; anyone from Silver Springs can come and dance. Because the women themselves have never been identified, no arrests have been made, and Maria is almost seen as a hero amongst many in town. Her long term ambitions are to raise a happy family, and to encourage other women to take up arms and live life, but for now Maria is just enjoying a new chapter in her own life.

\end{multicols}

\subsection{Residents of Note}

These people are some of the more notable movers and shakers in the city of Silver Springs. Not necessarily politicians or business owners, they nonetheless are powerful and influential characters, well known in the city.

\begin{multicols}{2}


\paragraph{Herbert Wyatt}
	Herb Wyatt is a heavyset man with a thick mustache and a deep unpleasant laugh, which is usually heard following his own jokes. He made a lot of money in the early days of the silver rush and now spends his money at saloons and brothels. Most people in town expect him to go bankrupt soon, with no obvious source of income and his blatant public spending, but it isn't quite that simple. 
	
He is deceptively well-educated and he doesn't spend his money as freely as people think; he merely doesn't bother hiding his vices. After amassing a small fortune prospecting in silver and copper, Wyatt decided to reinvent himself as a man of leisure. He received a medical degree from Tulane and a new identity from New Orleans. Upon returning to the area, he immediately invested in the saloon and brewing business instead of practicing medicine. He has a lot invested in the development of Silver Springs, and he will put money behind anything he thinks is worthwhile, especially scholarly projects such as the Readers.

\paragraph{Matron Madrigal}
Susan Madrigal is in charge of St. Francis Hospital. She was the head assistant in Dr. Campbell's office. With the most medical skill, short of the Dr. himself, she's a very capable and disciplined matron, and the hospital similarly seems to be successful and efficient. Ms. Madrigal cuts a very imposing figure. She is a stubborn, pale-skinned Scandinavian bear of a woman, and people have a tendency to behave politely whenever she's in the vicinity. She cares deeply for her patients and between the hospital and Church, she needs no other calling in life. Although she'll treat anyone, she does not hesitate to make it clear that she has strong opinions and is very distrustful of anyone who isn't a Protestant Christian.

\paragraph{Sheriff Paul Hammond}
Once a bright eyed recruit in a respectable east coast police department, an departmental spat earned Hammond a transfer out West, to the notorious and understaffed 
There, Hammond discovered that the Silver Springs \enquote{police department} consisted of a single jailhouse, privately funded by newspaper mogul Elias Reach, and a long list of previous sheriffs who'd fallen in the line of duty. Hammond's adjustment to his new situation was short, but brutal. Now several years into his career, his premature grey hair and deeply lined, unshaven face are souvenirs of his time keeping order in a city that resists his efforts at every turn. Though very much disliked by the criminal element of Silver Springs, Hammond is still grudgingly respected by many for his steady tenacity and dedication to justice.

Recently, he has taken under his wing a rag-tag band of recruits, each motivated by either obligation or a genuine desire to do good in their city. Though they're untrained, if even a sliver of Hammond's experience rubs off on them, they might just survive long enough to earn their keep.

\paragraph{Reverend Josiah G. Brightridge}
	Josiah Brightridge was a slave for most of his early life. After emancipation, he devoted himself to his church. His face is only beginning to show his age and life of hardship. 
Reverend Brightridge's sermons emphasize community and helping your fellow man. He came to Silver Springs only recently, invited by the local preacher Arne Caldwell when worries and questions about the Enochites began to affect the congregation. Upon his arrival he refused to be put up in Father Caldwell's home, preferring to sleep in the church. A charismatic speaker, he is ordinarily quite personable and calm, but has an intense distrust of anything supernatural. For this reason he also dislikes the Enochites, believing that their leadership is dabbling in practices and powers they should not be involved with. His willingness to hear people talk about their strange experiences outside the city has strengthened the church's following, but makes Father Caldwell uncomfortable. Brightridge's strong feelings about the supernatural lead many to believe that his past is filled with struggles against those kind of forces, but he tells no such tales.

\paragraph{Leon von Kaiser}
Leon von Kaiser can trace his family's lineage back to the lands beyond the Americas, where he claims that he was descended from kings. His father supposedly abdicated his throne during a period of revolutionary turmoil and fled to America. He bought and developed an estate at the foot of the mountains outside of Silver Springs and raised his only son there. While Kaiser's kingly pedigree is doubted, his father's name appears on the title for the estate, and that much is certain. His father was an intelligent and well-read astronomer, who bequeathed his library, observatory, and estate to his son Leon after his death. Leon von Kaiser sees himself as both the continuation of his family's bloodline and promoter of his father's culture.
Of the vast library in his estate, he has read very little, although he claims to have read it all. He sees himself as the West's intellectual because he is always ready to propose a new theory that explains everything. While he is largely dismissed by most serious American intellectuals as a radical monarchist, he has made friends in the local area who value his removed perspective on current events. 

Consumed by his dreams for the past and his obsession for the stars, he has given shelter to other radical thinkers and authors such as the infamous Coyote Teeth, who is rumored to be residing at his estate. In recent years, he has barred outsiders entrance to his estate - especially state bureaucrats - and installed a private rail line on his property that runs directly to Silver Springs. He has stationed personal guards to ensure that unauthorized people do not make it to his estate, and Mayor Woolcott has confirmed his von Kaiser's privacy. 
 Given that his most recent theory on the development of prosperous societies praises the use of indentured labor, the government of Silver Springs is concerned that he may have reinstituted slavery within his realm. There may soon be a reckoning with Kaiser, but even the Mayor is concerned about the loss of his library - the only bastion of knowledge in the West.

\paragraph{Jiao-Long Ma}
	Jiao-Long Ma was a Chinese immigrant and laborer for American Continental, working on railway construction sites for long hours and little pay. His overseer, Marshal Stotts, was a cruel, unforgiving man. To prevent himself from losing hope, Jiao-Long became determined to kill Stotts, planning to commit murder by crafting a gun from scraps of metal and stolen tools. For almost a year, he worked on his weapon in what little spare time he had, but when the gun was finished he realized that killing Stotts would simply lead to his own death. He hid his gun with the construction tools and abandoned his plan. No less than a week later, natives attacked the construction site. In the frey, Ma retrieved his weapon and killed three of them, saving his life as well as the life of Marshal Stotts. 
	
	In the aftermath of the attack, Stotts was in too much shock to think about why Ma might have been creating the gun in the first place; he only realized that Ma's genius was being wasted hammering railroad spikes. He offered Ma a business proposition, they would go into business crafting and selling guns, Stotts would handle the front end and Ma would do the manufacturing. Their partnership was short-lived, Stotts was killed in a duel and Ma's business lost it's face. But his guns still show up around the West, each a work of art, no two alike. Every gun he's made has a reputation of its own, that can be followed through the west, through the hands of some of the most skilled and influential people in the West. New Jiao-Long Ma firearms don't show up often but when they do it's always in the hands of a larger-than life character looking to make waves, and no attempts to figure out where they came from before that person's possession ever bear fruit.

\paragraph{Gertrude H.  McCormick}
Mrs. McCormick is the local school teacher, the only school teacher, in Silver Springs. She's a tall, skinny, plain-looking woman who always wears a scowl, glasses, and her long brown hair in a braid almost all the way down her back. One of the only people in town to have been born on the frontier, in what was then the Turage Territory, she believes that there's a future for Silver Springs, and it lies in the children. There are few children in Silver Springs, and Mrs. McCormick is determined to make sure that they have good schooling and good Protestant values so they grow up to be contributing members of society. For these reasons she also sits at the head of the Silver Springs vigilante council, and she is the only woman to sit on the council at all. In her opinion, the drunks and criminals that move in and around town are the biggest threat to the children's future; they are dangerous and provide terrible role-models. She campaigns to rid the city of alcoholics, and as one of the best rifles in the state, she is very comfortable picking up arms and tracking down the various bandit groups that harass the citizens and shamefully inspire the children.

\end{multicols}

\section{Government and Army}
The election of President Manford Silas Wolfe and the ongoing campaign of General Steele matter to the folks in Silver Springs. These two men rose to prominence during the civil war as generals, and while Wolfe is trying to repair the country after the bloodshed of the civil war, Steele hasn't truly left it. 

\begin{multicols}{2}


\paragraph{President Manford Silas Wolfe}

Wolfe is a controversial figure. He rose to prominence as a general during the Civil War in which he was renowned for breaking the stalemate between the north and south through bloody and costly offensives. A general for the north, he directly led the campaign which decimated the southern army. His campaign for the presidency elicited fear amongst the defeated south, who cried in outrage that his election would entail further punishment for the southern states. So far, Wolfe has adopted a policy of forceful reconciliation and reconstruction, opening up the west to migration and statehood, while rebuilding the south's devastated infrastructure. 

He is relentless in forging a consensus - demanding personal duels from those who oppose him. His bloodstained reputation followed him into office, where many have adopted the phrase, \enquote{Man is a Wolfe to Man.} There are even rumors that President Wolfe is actually a werewolf. Although he has publicly laughed at the suggestion, it has become a popular metaphor when discussing the man. After all, who could endure this much carnage, yet still remain human enough to reconstruct a nation beset by the goriest civil war in history? The rise of Silver Springs has been largely attributed to Wolfe's success, as he has staked his power upon the loyalty of new, local leaders in the West. Mayor Woolcott has taken a particular liking to Wolfe, seeing him as a beneficiary and friend, although Wolfe is much cooler towards him. 	

\paragraph{General Adrian Steele}
Steele was born to a family of farmers on the east coast and grew-up with a relative abundance of everything. He developed traits of both stoic belligerence and impetuousness that would define his later career. Steele enlisted at the outbreak of the Civil War and quickly rose through the ranks amongst the cavalry. He employed devastating counter-offensives and magnificent feints; his techniques were noted for their ruthlessness. 
	
After the war, Steele found gainful employment in Wolfe's programs of Westward expansion and his reputation skyrocketed. His most discussed battle occurred at the foot of the mountains by Silver Springs, in which he encircled the Confederacy of the Hawk's Grand Chieftain, Tecumseh, and eradicated him alongside his cohort. Steele's name is poison to the Grand Chieftain's successor, the Starhawk Kenu, who has vowed to slay him. A number of skirmishes over Van Wyke's passengers trains have occurred between Steele and Tecumseh. Steele uses these trains as bait, drawing the Confederacy into a committed military engagement, and then overwhelms them in a counter-charge, but he has won no decisive victory so far . Steele is the number one enemy of the Confederacy of the Hawk, and Tecumseh has matched him battle for bloody battle. To ensure his position, Tecumseh needs unambiguous victories and that is precisely what Steele denies him. 

Finally, Steele's promises serve as a rallying cry for restless war veterans, who re-enlist in multitudes. He offers them a break from their shattered lives, a temporary, new life laden with adventure and spoils. When the west is settled, Steele knows that his name will be ranked amongst the heroes who carried the sword and defeated the last bastion of resistance.  
Needless to say, for all Steele's fantasies, Sylvester Ernest Van Wike has not been too receptive about his passenger trains being used for the military's goals.

\end{multicols}


\section{Native Americans}

\begin{displayquote}
\textit{It is important to note that this section on Native Americans is entirely fictional. This is intentional, as it was the designer's intention to provide a representation of Native American life that encompassed the larger issues they faced, rather than the issues specific to natives of the area that is now the State of Colorado. There was no single religion which every native tribe adhered to. This fictional representation is heavily inspired by the Jicarilla Apache Texts, which are available online for free.}

\textit{The indigenous people have a very harsh life, and there are no signs that it will improve. For the past four centuries, these people have contended with a population of foreigners encroaching upon their ancient territory. With guns, railroads, disease, and legislation, Native Americans were corralled or pushed westwards.}

\textit{As the sun sets in the West, millions of Native Americans find themselves caught in the twilight. This is their story, kept neatly between these pages, though it is entirely fictional. It would not have been possible to accurately define Native American life within the confines of this book. }

\textit{This book is a work of historical fantasy. Any pretense of historicity would only further the inaccuracies which surround representations of Native Americans in popular culture. The most recent Lone Ranger (2013) film is a good example.  Even the translator is a kind of Conquistador, despite their best intentions to inform. If you are playing this game while speaking English on the American continent, you should deeply consider why this is the case. That is the sole intent of this story.}

\textit{This is a story - and it is certainly not a history - of the Native Americans from their first days until 1890 when the Frontier was officially declared closed.}
 \end{displayquote}
 
 
\subsection{The Beginning}

\begin{multicols}{2}

	In the beginning, there was only the Great Spirit. The Great Spirit made the world, drew forth the rivers, and populated the land with innumerable creatures. Amongst these creatures, the Fox was the smartest, and he fooled the other animals into making him their chieftain. He soon became bored, and beseeched the Great Spirit to entertain him. The Great Spirit, seeing that the Fox had mastered all of creation with his guile, created humanity. The Great Spirit gave the humans the gift of wisdom, so that they could spot the trickery of the Fox. Over time, this wisdom became paranoia, and the humans dreaded the approach of the Fox, the being who stalked the line between falsity and reality. The humans asked the Great Spirit to protect further against the machinations of the Fox. 

The Great Spirit agreed and created three birds: the Raven, the Vulture, and the Magpie, who were designed to live in harmony with one another. So long as these birds lived, the Fox would not be able to fool all of the humans; their priests would always be able to discern the  truth from the Fox's lies. If, however, the birds were to die, the humans would also die. If creation were to end, the birds would eat each other, and thus the birds became known as the Three Cannibal Birds. The priestly orders would keep the birds at peace with one another, resolving any conflicts before they resulted in death. The Three Cannibal Birds represented a covenant between the Great Spirit and humanity, so that the Fox's existential humor was kept at bay. But the Fox still played with the humans, leading them into minor follies and tricking them into personal tragedies. The Fox had been misunderstood by the humans - he was not malevolent, merely bored. He watched as the simple humans, who so prided themselves on deep insight, engaged in the most ludicrous acts of self-deception. They did not need the Fox to deceive themselves, because they already took so much pleasure in their innate duplicity.

The Fox loved the humans more than anything, but the humans feared the Fox. The Fox laughed at their fear, but the humans were right to fear the Fox who would unwittingly bring about their demise.

\paragraph{How the Pox Tricked the Fox}
One day, while the Fox lay upon the eastern shores of the continent, boats began to appear on the horizon. The Fox, sensing the opportunity for mischief, informed the local tribe of their arrival. He dazzled them with stories of their foreign wonders and terrified them with descriptions of their weapons. The Fox tried to portray these foreigners as divine and powerful, but the tribe was wiser than the Fox was cunning. They knew that history was unfolding before them, and they waited for the foreigners to make the first gesture. 

When the ships arrived, there was rejoicing and wonder on all sides, yet each people cultivated a suspicion of the other. The Fox saw that these two peoples were fated to be incompatible, regardless of their innocent intentions and conciliatory behavior. The Fox doubted that their altruism would last and the lingering gunships reminded the Fox that his people would die if the two went to war.
 
As night came, the tribes were struck with a devastating plague. Millions of tiny pustules covered their bodies, and the sound of vomiting blotted out the foreigner's songs of friendship. 

\enquote{This is the beginning of the end!} The Fox cried and sang a funeral dirge as the coastal people fell in droves to the new disease. For a whole week, the Fox wandered the makeshift graveyards, completely bereft. He tried to escape the corpses piling around him, but wherever he walked, the disease followed.
 
He called the Raven to him, and told the Raven to warn his people about the foreigners and their disease. This, however, was the moment when the pox tricked the Fox. The Fox had carried with him the disease, but was unaffected by it, so that he spread the pox as he walked. Now, the Raven had become infected. The Raven flew to warn the other tribes of the continent, and as he flew, he unknowingly spread the plague. Once the Raven warned all of the Chieftains about the foreigners, he collapsed.

The Vulture was the first to realize the tragedy of the Raven, who had finally succumbed to his illness. On the Raven's deathbed, the ground blackened beneath him, as pustules formed upon the land and the Raven heard the earth's breathing become labored. The earth sighed with quakes and became even more ill. The Vulture came to the side of the Raven, and repeated the promise made by the Great Spirit to the humans - that one bird consuming another would lead to the end of the world. The Vulture promised never to eat another living bird, so he urged the Raven to die.  If the Raven died, the Vulture believed that he could eat the Raven and prevent the poisoning of the earth without heralding the end times.
 
The Raven wept, and told the Vulture that he was cursed with immortality - it was impossible for him to die. In order to save the land, the Vulture must consume the Raven alive and thus break the covenant between the Great Spirit and the humans. In the interest of saving his people, the Vulture consumed the Raven. 

The Magpie watched, and felt overwhelming sympathy for humanity's paranoia. They had been right about the Fox, and now the end times were among them. As the years passed, he watched with deathless vigilance as his people were slaughtered by disease, despair, and desolation.  The Vulture watched the Magpie watching him, and felt a fraction of the melancholy which the Fox must have felt when humanity turned away from him. At first, the Magpie aimed to prevent the apocalypse, but over time, his disposition shifted. It is only recently that many people, following the apocrypha of Gizzard Tongue, believe that the Magpie has finally lost his mind.

\end{multicols}

\subsection{Civilized Five}
The Civilized Five are the remnants of the five tribes in the Silver Springs area who are currently being assimilated into American society. While they live almost exclusively on reservations, there are exceptional individuals who brave the dangers of leaving these reserved locations. Despite their status as \enquote{civilized,} they are almost unanimously regarded as savages. They take this nickname with a dark sense of irony, as they are violently aware that their own civilization has been eradicated. To the American colonists, civility is equated with docility, and the name of their tribal federation is proof enough that the Trickster Fox has paid too much attention to them. The critics of the Civilized Five point to the poverty of their estates and blame the tribes for retaining their archaic lifestyle. The tribes also have their staunch defenders in American society; those who demand that the Civilized Five receive more comprehensive aid from the government in their time of great need. 

Perhaps the most (in)famous member of the Civilized Five is Coyote Teeth, the brilliant satirist. Unsurprisingly, her books, written entirely in English, have been widely popular and showcase the deplorable living conditions faced by her people. However, her reputation is complex and soured with the kind of mischief one would expect from a devotee of the Trickster Fox. She is the public image of the Civilized Five, for better or for worse. 

\begin{multicols}{2}

\paragraph{Background}
Initially, when settlers tried to create a colony on the Jefferson Stripe, the riverbank near Silver Springs, they were evicted by the Five Tribes who lived there. 

Years later, as the West opened up, a second wave of settlers aimed for the river of Silver Springs. They were cut off by the five tribes and forced to leave the territory. Seizing this opportunity, the army gave Adrian Steele the order to crush the tribes. During the night, Steele surrounded them. When day came, the tribes attempted a breakout, but their charge was broken upon the bayonets of Steele's troops. To this day, the chronicles speak of Steele's brutality as he sacked the tribal camps.   The five tribes fled to the river they had defended, and there they surrendered. Steele forced the tribal chieftains to abdicate their sovereignty and divided the tribes into five different reservations throughout the area. 
A generation passed in silence and depression. Revolts were common, and so was suicide. The priests of the tribes have adapted their traditions for the present age, and they serve as the only conduit to the ancient traditions from which their peoples have so radically been severed. 

\paragraph{Religion}
To the Civilized Five, their religion explains the history of their people - their movement from a once-sovereign people to desolation and isolation. After the abdication of their chieftains at Steele's hands, the priests of the Civilized Five have become the sages of culture and tradition. They provide what moral support they can, but lately this has not been enough. Nihilism is rampant amongst the camps of the Civilized Five, and even the priests are uncertain whether their stories and rituals provide any relief to their broken people. They continue the sacred practices, but their reverence has become mild. Their people do not fear the Fox as they used to, as if they have already accepted their defeat.

In a dream, a priest named Gizzard Tongue watched as the Magpie circled the continent, trying to find any tribes who had successfully resisted the foreigners. When the Magpie arrived on the West Coast, he saw the warships being constructed in California's harbors. While the Fox despaired when he saw gunships made of wood, these new ships were made of steel. The Magpie flew into a rage when he saw that the foreigners had finally won- it was as if their ships sailed through the continent and emerged stronger upon the other shore. Now, these foreigners would be free to export their terror across the world. Something must be done about this, the Magpie thought. If the end of the world was brought about, at least it would halt the progress of these foreigners, the progress they viewed as a god, the progress of finally colonizing a continent, progress like Americanization, progress like ironclad gunships sailing to foreign shores. All this progress was white noise to the Magpie, whose own thoughts progressed rapidly towards a final, prophesied conclusion- the Great Spirit's promise to end everything. 

The Magpie, the smallest and gentlest of the Cannibal Birds, sought out the Vulture and tried to free the Raven that was in his stomach. The Vulture told the Magpie that it was no use, that the Americans were immune to the plague that they brought upon us. This did not satisfy the Magpie, whose only desire was the end of all creation. He attacked the Vulture, tore open his stomach, and ate both the Vulture and the Raven decaying in his stomach. This was the vision Gizzard Tongue had.
 
The story spread like wildfire throughout the camps of the Civilized Five and many Priests of the Vulture renounced their profession. A council of priests was held, and it was decided that the Trickster Fox should be revered above all, for now their people were truly humbled by the vicissitudes of fate. Salvation was impossible - they could only laugh at the tragedy of their people, foretold since the earliest of times, and they laughed at how they had failed to prevent such an obscure prophecy from becoming reality. 
If there is any role that religion plays in civilized society, it is the injunction to laugh at the insurmountable sorrows facing a people before their inevitable silencing. Whether this divine comedy is secretly couched in nihilism or in a defiant repudiation of it, the Trickster Fox stands alone before the Civilized Five: terrifying and hilarious. 

\paragraph{Coyote Teeth}
Coyote Teeth was raised on a reservation and learned to read and write at a young age. She left the reservation and made friends with Leon von Kaiser, whose library she passionately enjoyed and devoured. Von Kaiser subsequently became her patron, and she embarked upon a career as a ghostwriter, adopting the name \enquote{Coyote Teeth.} She wrote bitter satire about the conditions her native tribe suffered on the reservations, employing a grim humor that was said, \enquote{to sound like the noise a coyote makes when laughing through its teeth.} She rose to prominence amongst the educated Eastern classes, who read her magnum opus, \enquote{Man is a Wolfe to Man}- a satire on the current President Wolfe. In the novel, the protagonist Ulysses S. Grant is a thinly veiled reference to Manford Silas Wolfe. It follows his role through the recent civil war, detailing his atrocities and providing absurd justifications for them. For example, there is a running joke throughout the text that Grant's (and therefore Wolfe's) violent tendencies are due to his secret nature as a werewolf - a notion which became extremely popular following Coyote Teeth's prank.

One day, when Wolfe was visiting Silver Springs on parade, Coyote Teeth, with the help of Leon Kaiser, captured over 2 dozen wolves and set them loose down the avenue. What seemed like a joke turned serious when the procession of wolves tore into the crowd. Wolfe personally killed 3 of the wolves with his bare hands before his guards shot the rest. At the end of the skirmish, over 20 people were injured, but Wolfe was jovial and the crowd delighted with the martial prowess of their warlord-president. Their cries of \enquote{Wolfe is a man of wolves!} echoed throughout the city for days. Mayor Woolcott quickly discovered Coyote Teeth and Kaiser's involvement in the matter, but he has been reluctant to pursue them for other reasons.

In the past year, Coyote Teeth has caused even more controversy, writing a series of tragic-comedies about the lives of natives on reservations. While incomparably witty, her novels depict the grim realities of life on these reservations. Starvation is common, disease is rife, and culture is prohibited. This has divided a number of intellectuals on the emancipation of the natives, and the debate rages to this day. The Trickster Fox plays a key role in these tragic-comedies. Coyote Teeth's reverence for the Trickster Fox is absolute. She views herself as the Fox's emissary, and maintains close ties with the Civilized Five, who see her as their champion.

The recent von Kaiser controversy casts shadows over Coyote Teeth's whereabouts. It's more than likely that she's fled von Kaiser's domain, but many are concerned that something much more sinister has happened; that she realized too great a comedy, laughed too hard for too long, and was changed forever. These speculators are concerned that another tragedy has occurred- that Coyote Teeth, like von Kaiser, has realized something deeper and darker about the human condition, and she laughed at it. This is, of course, just speculation; speculation that is perhaps mixed with the desires of those who want her criminalized and her books banned. 

\end{multicols}


\subsection{Confederacy of the Hawk}
The Confederacy of the Hawk is an aggregate tribal organization composed of numerous tribes fleeing the encroachment of the United States. Led by the Starhawk Kenu, the Confederacy is the largest alliance of Native American tribes to date.  They operate under the protection of the mountains west of Silver Springs, the Argent Range, and have surprisingly peaceful relations with the local Enochites. The Confederacy of the Hawk are committed to the reclamation of all their lands, by force, however, their alliance is anything but permanent.

The natives have been deeply involved in the wars fought in the territory. They have served on both sides of nearly every conflict. This has divided them and they remain divided to this day. Just as the Confederacy of the Hawk has been divided from the Civilized Five, the chieftains remain divided over the status of the Cannibal Birds. While they reject Gizzard Tongue's dream (see Civilized Five, Religion), the priests of the Confederacy of the Hawk have not yet agreed upon the state of these birds. Kenu has proclaimed that this minutiae is irrelevant to the present state of the Confederacy, and has indefinitely suspended any discussion of the matter. For Kenu, all that matters is that the end times are upon his people. He sees these petty theological debates as meaningless and prone to divide the confederacy, which must remain united at all costs. While his subordinates disagree, they do so silently, afraid of incurring his wrath.

The Starhawk's largest concern is supplying his people with the necessities to survive.  With the recent tribal migrations, the land is unable to provide for all of its inhabitants. As a result, the Confederacy of the Hawk has committed their resources to robbing the railroads of Silver Springs for survival. 

These railroads are well guarded by General Adrian Steele's cavalry, which inflicts brutal losses upon the Confederacy. Things are looking grim for the Confederacy, and many tribal leaders are resigned to their fate. The Starhawk speaks of one last great charge against the United States - to settle the issue of the Silver Springs once and for all. 

\begin{multicols}{2}

\paragraph{Background}
The Confederacy of the Hawk was born from a divergent prophecy in a nearby land (roughly the Lehigh territory). The most widely accepted form of the prophecy states that, when the covenant between the Great Spirit and the humans is broken, the Great Spirit will give the humans a fourth bird, the Hawk, to govern humanity in their final days. The Hawk's presence on earth will be taken up by the Starhawk, the individual whom the priests of the other three birds will unanimously nominate. The Starhawk will unite the scattered tribes underneath a single banner, so that their people will at least have solidarity in their final hours. 

The Confederacy of the Hawk didn't begin as such. The tribes which compose the confederacy, led by the Grand Chieftain Tecumseh, migrated into the plains behind the Argent Range once they were beaten by the American military and routed. Their defeat was reported to the nearby tribes in the Deseret territory, who flocked to their leadership. The Grand Chieftain Tecumseh was unanimously chosen to lead this confederacy of Native American tribes. 

While Tecumseh promised to correct his previous defeat, he did so in increasingly religious language. He spoke about the coming of the Hawk, the bird which was to arrive once the covenant was broken, and pointed to his people's current allegiance as a sign of the Hawk's arrival. 

The priests of the Confederacy convened to discuss Tecumseh. The tribes from Tecumseh's land insisted that Tecumseh was the prophesied Starhawk,  but the rest were skeptical. Because the proclamation of the Starhawk required a unanimous consensus amongst the priests, the meeting was adjourned in frustration.  

The majority of the tribes were inflamed by Tecumseh's rhetoric and demanded a war of revenge against the United States. Tecumseh gave them their war, and led a mass offensive to reclaim Silver Springs. Their advance led them to the city's gates before they met any serious resistance. General Steele and his army met them, deployed around the city limits while his cavalry rested on an overlooking hill. 

Tecumseh's host paused. He split his forces to affect a spearhead which was designed to pierce through Steele's lines and envelop his infantry from behind. Tecumseh planned for Steele's cavalry to become too mired amongst their own troops to effectively charge, thus blunting Steele's proposed counter-attack with his cavalry. He would personally lead the charge as the spearpoint, and drive his people to victory.

Steele, however, had a trick up his sleeve. He was secretly accompanied by two brigades of his finest cavalry, garrisoned inside Silver Springs. Steele placed the town under martial law, and ordered his officers to alert him once Tecumseh began his charge. 

Tecumseh's charge is one of the great tragedies of history. His spearhead pierced through Steele's lines, but were blunted by the counter-charge of Steele's cavalry. Tecumseh and Steele met at the gates of Silver Springs, where Tecumseh was struck down and his charge halted. To this day, the spot where Tecumseh fell is a landmark of Silver Springs, called Tecumseh's Gate, where a statue of Adrian Steele brandishing a broken spear was erected.

Once Steele's cavalry was engaged, the other cavalry on the hill joined the fray, and broke through the shaft of Tecumseh's spear. The spearhead, including Tecumseh's top lieutenants, was swiftly consumed and his army broke. 

It was a total rout, and the Confederacy of the Hawk fled back to their camp behind the Argent Mountains. They had lost nearly half of their men. 

\paragraph{Starhawk Kenu}
When the warband returned to their encampment, the confederacy spent an entire week in grief. During this time, the chieftains of the various tribes met to discuss the future of the confederacy. Tecumseh had never specified his successor, and the chieftains were concerned that the confederacy would be disbanded and scattered to the winds. After much deliberation and disagreement, they nominated a youthful Priest of the Magpie named Kenu as their new Grand Chieftain. 

Kenu's grandfather was a chieftain for one of the Civilized Five's tribes who were forced to abdicate after their military defeat. (See Civilized Five, Background) Kenu bore this legacy proudly, claiming that he alone spoke for the entire Civilized Five. While the claim was seen as childish and arrogant by the council of chieftains, they saw in it a powerful logic. If they proclaimed Kenu to be the Starhawk, they could claim to exercise their authority over the Civilized Five, whom Kenu claimed to represent. 

When the chieftains convened their next council, they invited Kenu and the remnants of the confederate priests to attend. The chieftains rose before them, and proclaimed that Kenu was the Starhawk. The priests were outraged, and unanimously threatened to abdicate their office. 

Kenu, however, had been informed of his ascension beforehand, and used the time to prepare a speech. He explained his pedigree to the priests, detailing his claim of command over the Civilized Five. He proposed that, above all else, his entrapped people must be liberated and that, once liberated, they would join the confederacy to replace the losses from Tecumseh's charge. He claimed that Tecumseh was not the Starhawk because he had failed to consider that his brothers in chains, the Civilized Five, should also live under the sovereignty of the Confederacy of the Hawk. While their religious practices did not mention the Hawk, that was no fault of their own. Because they observed the same basic traditions, Kenu argued that they should be liberated as well, and brought into the fold.

The priests secretly knew of the nihilism which beset the Civilized Five, and knew that their leaderless people would be hesitant to exchange one master for another. While the priests remained in disagreement over the proclamation of the Starhawk, the council of chieftains did not permit the priests to convene their own council to adjudicate on the matter of the Starhawk. It became immediately obvious to the priests- this was a coup. 

The Starhawk was proclaimed, and any dissenting priests were exiled. An official ceremony was held shortly afterwards, in which the remaining priests unanimously affirmed Kenu as the Starhawk. There was no great spectacle for Kenu's coronation. The ravages of hunger and malnutrition were felt throughout the camp, as the land showed that it could not feed the confederacy indefinitely. 

The Starhawk Kenu, only 16 years old and filled with that particular youthful arrogance towards death, personally organized a series of raids upon the trains which criss-crossed the frontier. Nearly every attempt to loot the trains was repulsed, once again, by Adrian Steele. Kenu and the chieftains were infuriated, and their anger towards the United States deepened as their people starved.  

Kenu can no longer justify these small-scale raids, and his chieftains have suggested that he should make good on his grandfather's claim. Kenu knows that he will have difficulty bringing the Civilized Five into the confederacy and plans to use force. He has been meeting regularly with the council of chieftains and in the deepest secrecy. It is unknown what Kenu is planning, but his tribesmen believe that Kenu is preparing for the end. If Kenu is going to launch a final offensive against Silver Springs, he will undoubtedly need the support of the Civilized Five. The movement of his troops indicates that this liberation is his next goal, whether or not the Civilized Five like it. 

\end{multicols}


\section{The Church of Enoch}

A religious organization unique to the American West, the Church of Enoch is both spiritual movement and economic doctrine. Their unorthodox beliefs lead to the persecution of Enochites in most settled lands, leading to their self-imposed exile to the frontier. Ever seeking to expand their influence, Enochites have established their own state of Deseret to the West of Jefferson state. While they nominally follow the requirements laid out for them by the Federal government, many fear the Church is seeking to establish a new Theocracy the wastes. 

\begin{multicols}{2}

\paragraph{History}
Founded by Hyrum Cutler in the first quarter of the 19th century, the Church of Enoch has faced persecution and mistrust since their original exodus away from the East coast. After settling and subsequently being forcibly ejected from several states, including New York, Indiana, and Illinois, the Church relocated to the US territory of Utah. There, they were able to live unmolested for several years.

Members of The Church of Enoch believe that North America is home to the original location of the garden of Eden, and that Native Americans are a lost tribe of Israel. Central to the beliefs of Enochites are the United Order. Adherents to the United Order would willingly cede surplus production to the church, which is distributed by a bishop or equivalent religious leader. Additionally, Enochites practice polygamy, where family patriarchs would take multiple wives. These traits together, along with their powerful economic presence and tendency to vote in blocs, made them a threatening force in the eyes of more traditional settlers. Violent conflict between Enochites and \enquote{gentile} settlers was not uncommon before their move to the Utah territory, nor was violence between schism groups that split from the mainstream church. In response, church doctrine developed a \enquote{kill-or-be-killed} approach to attacks on church members, and members were not unopposed to acts of terrorism or preemptive strikes against their perceived enemies. This attitude was only reinvigorated by the murder of Hyrum Cutler by vigilantes as the Church fled Illinois. In spite of this, however, most Enochites are reserved and devout, truly believe in the righteousness of the United Order, and freely welcome even the most unscrupulous of strangers, so long as they do not threaten violence against the Church.

Founding an entirely Enochite government under the leadership of Cutler's appointed successor, Theron Marsh, the Enochites settled in the uninhabited salt flats of Utah territory. Desiring a form of representation in the United States government, as territory representatives are appointed rather than elected, Church leaders drafted a constitution for the state of Deseret, and submitted it to Congress. Eager to avoid conflict, and desiring another abolitionist state in the strained years preceding the Civil War, then-President Zachary Tyler ratified the constitution in spite of resistance from other members of the government. Despite fears over what precedent might be set by allowing the church to form their own state, Deseret was founded. More recently, however, a disastrous incursion into the state of Deseret by the Federal government, misinterpreted by the Enochites as an attempt at a forcible retaking of the state, led to open conflict. In the months following, Enochite militias clashed with underprepared Army forces across Deseret. A truce was eventually negotiated, and though the Deseret government is now no longer entirely Enochite, Theron Marsh, who ordered the original attack on the Federal task force, was allowed to keep his position as the head of the Church.

Currently, though tensions run high between the Church and the Federal government, there is an uneasy peace. However, schismatic churches, led by newly declared prophets who see the fight with the United States as a holy war, have sprung up out of the chaos. Though the official stance of the Church of Enoch is that these groups are heretical, intervention is rare. Though Enochites are finding themselves more and more disqualified from holding occupations outside of their home State, missionaries are a common sight on the streets of Silver Springs, as Jefferson borders the state of Deseret to the East. 
	
Rumors abound that Theron Marsh, the highest authority in the Church, has been receiving instructions from some other power. Not helping this is the fact that he has been known to retreat for long periods of time into a shrine behind his study, ornamented with only a strangely carved stone laid carefully in a velvet-lined case, before emerging with new declarations and decrees. While certain Enochite sects believe that great people in the Church will be reincarnated, inquiries as to whether Hyrum Cutler's spirit may still be bound to this earthly realm are not taken well by Enochite leaders, and may even inspire violence from less tactful adherents to the faith.
 
\paragraph{Theron Marsh}
Theron Marsh is a firebrand leader and a man with the needs of his people in his heart. Biblical in his nature and his appearance, Marsh's long, greyed beard and simple dress do little to hide the fire behind his piercing, deep blue eyes. 

Before his time as Church president, and even before he earned a place in the Church leadership, Marsh spent time in Federal prison, arrested on charges of destroying the homes and property of dozens of people who had actively harassed and denied service to Enochites living in the outskirts of a major city. Throughout the court proceedings, Marsh insisted his innocence, claiming it was \enquote{the hand of God} that brought on the fire he was accused of starting. Though there was not enough evidence to convict him of arson, a number of other offenses, the least of which included his admitted polygamy, were enough to put him behind bars for five years.

On his release, Marsh discovered that during his imprisonment, Church leaders had been championing him as a martyr for Federal oppression of Enochites. Embracing his new role as an icon among his people, Marsh began a tooth-and-nail ascent through the bishopry. Though his rhetoric of retaliation against the people who sought to harm the Church was a drastic turn from earlier doctrine, it found supporters on all levels of the Church hierarchy.

When Hyrum Cutler, the church's first prophet, was killed in the chaos of the Enochite's exile from Illinois, Marsh took up the reins of leadership with little resistance by Church elders. It was Marsh who led the Church to the salt flats west of Jefferson, and Marsh who spearheaded the campaign to give statehood to the sun-bleached territory there. 

Marsh's first Proclamation, a message he claimed was interpreted directly from the word of God, came at the end of the chaotic and violent Deseret War. A short but bloody conflict between Enochite militias and the US Army, Marsh called for a truce he declared was God's will for his people.

What Enochite elders are less willing to share is that this announcement came after Marsh locked himself in his study, refusing all food and water for three days. Upon exiting, he disclosed in confidence to a handful of elders that it was not God advising him, but the late Hyrum Cutler. A fist sized rock, perpetually warm and inscribed with arcane glyphs, which Marsh cannot explain how he came to possess, serves as Cutler's new earthly form, which he claims to be able to telepathically communicate with.

Marsh's later Declarations, including ending the practice of Enochite polygamy, have led to major changes in church doctrine, inspiring dissent from some of his followers. Insisting that these changes are the word of God, Marsh and the rest of the Church leaders have not been kind to these seditionists. 
 
\paragraph{John Caleb Sprigg}
	John Caleb Sprigg was the son of prominent Enochite Pastor Jeremiah Sprigg and raised in his father's shadow. Father Sprigg, as the leader of his United Order, was constantly keeping track of budgets, inventories and the full economic production of his community. John, often invited to observe his father's work, spent his younger years being molded to follow in the Elder Sprigg's footsteps. On a mission to the East Coast in his later years, however, he found his true love, a young woman by the name of Julietta, and his true calling: the railroad. Inspired by the miles and miles of rail and the massive, steam-powered engine cars, he saw visions of an Enochite train bringing the entire Church together to and from a single community. Upon finding that the rest of his hometown had little enthusiasm for his high-minded dreams, at the urging of his wife he founded his own rail company, the Deseret railway, with funds provided by his sympathetic father and an unexpected windfall from his in-laws. At first only a local rail company, with tracks that barely found their way out of the state of Deseret, they soon expanded beyond the horizons of the single state.
	
Now the Jefferson Deseret San Francisco Railway, Sprigg's rail empire is the second largest in the West, preceded only by the monolithic American Continental. Long gone are Sprigg's visions of a church united by train, as the JDSF rail snakes its way across the west coast, joining cities that have only meager Enochite presences with the devout communities of Deseret. While Sprigg continues running his business at the behest of his wife, his true passion is using the incredible potential of the railroad to bring prosperity to his people.

\paragraph{Julietta Clements Sprigg}
The daughter of a New York financier, Julietta Clements was raised in a well-to-do home and worked as a desk clerk for a news agency when she met John, an Enochite missionary. John was a true believer in his faith and sought to find new followers willing to make the pilgrimage back to his home state of Deseret. 
In Julietta's case, it worked better than expected. After a series of coincidental encounters, one stymied mugging, a street-vendor lunch date, and a longer-than-expected train ride, the two of them fell in love in the way only storybooks can tell it. Wasting no time, they were married less than three months after their first fateful meeting. In her, John saw God's love for mankind. In him, Julietta saw a passion unbound by station or circumstance.

Upon arriving in the desert country that the Church of Enoch called home, however, she was distressed to find that John's tales of a land of prosperity, equality, and faith were not quite true. John's family lived in a small community separated from the Enochite capital by both distance and culture, and as she was not a member of the church, her and John's marriage was deemed unacceptable. For some time, the two of them lived as social pariahs, present in the community, but unwelcomed by it. Undaunted, Julietta put her clerical skills to work under John's father, one of the few who welcomed the young woman to their town.

Though it took years to even admit it to herself, she couldn't stand it. Certain it was her best path to salvation, she took drastic action. Calling in a favor from home, and convincing the elder Mr. Sprigg to provide financial support, she and John founded the Deseret railway. Although Julietta still claims that her love for her husband is as endless as it was on the day of that first train ride, circumstances have changed, and she along with them. The JDSF railway is the second largest in the West, and is rapidly closing that gap. John's passion has faded as his vision of an Enochite community united by rail has given way to the reality of running an empire, but Julietta's has only surged. Though John is nominally the head of the company, it is Julietta who is the fire fueling the JDSF railway's rise to power.

\end{multicols}