%----------------------------------------------------------------------------------------
%	SHORTHAND RULES
%----------------------------------------------------------------------------------------
\documentclass[../main.tex]{subfiles}

\chapterimage{Great_Plains_1x2.jpg} % Chapter heading image

\chapter{What is this game?}

\lettrine[lines=2]{A}{ny} tabletop roleplaying game is an exercise in group storytelling. A Game Master and a group of players come together, and with some imagination, acting, strategizing, calculating, ignoring risks, making stuff up, overthinking things, takeout food, bad luck, good luck, scheduling, and yes, a little bit of math, everyone gets the chance to not just tell a story, but to be part of a story. In keeping with that, every role playing game has a type of story that they're best at helping you explore. 

American Sunset, the game you're holding in your hands now, is a game about the West.

When we set out to make a Western role playing game, we knew we wanted it to feel like a Western. A world where death is cheap, and living is hard. Where the open frontier stretched on in every direction, and mysteries and adventure are always just over the horizon. Where maps aren't always complete, and where tales of impossible feats and stranger things are true as long as the stories get told.

American Sunset is quick, tense, and dangerous. Combat is decisive, and magic, while real, is distant and always a hair's breadth away from discovery. Player characters begin the game with well-established identities, capable of holding their own against the forces of the frontier. By the end of their journeys, those same characters have become more than the sum of their parts, emblazoned on the world as more myth than history, more tall tale than documented fact. Similarly, the environment of American Sunset is designed to give players and GMs a complex, fully-realized landscape to explore. The city of Silver Springs sits at the center of multiple factions eager to see their vision of the West realized, and at the crossroads of classic American vistas. The frontier itself is populated with a diaspora of peoples, some displaced, others well situated, but all making a living on the land, any way they can.

We hope that in playing American Sunset, you have fun, come to see the West from a new perspective, and most importantly, get the chance to be a part of a Western tale all your own.
\vfill
\section{Shorthand Rules}

\paragraph{How to Perform a Contest}

(For a detailed explanation see \textbf{Contests} pg. \pageref{Contests}

The only dice ever used are six sided dice, and the vast majority of contests use two per character. The total of a roll is the numbers rolled on two six-sided dice plus the rating of the relevant skill being used.

\hfill

In an \textbf{Unopposed Contest}, characters roll against a Target Number.

In an \textbf{Opposed Contest}, two characters roll, and the higher total wins.

\hfill

In all contests, the procedure is as follows:
\begin{enumerate}
\item Active character sets the stakes.
\item Defending character (if there is one) is allowed to raise the stakes.
\item Roll dice.
\item Calculate and compare totals.
\item Resolve the contest.
\end{enumerate}

\hfill

When rolling at \textbf{Advantage}, players roll four dice, and choose the top two values.

When rolling at \textbf{Disadvantage}, roll four dice, and choose the lower two values.

\hfill

\textbf{Traveling} is a unique type of Unopposed Contest, and is determined by performing three different Contests: Awareness, Fitness or Riding, and Survival skill, each with their own target numbers and consequences for failure.

\paragraph{Reputation} 

(For a detailed explanation, see \textbf{Reputation Points} pg. \pageref{Reputation Points})
Players are allowed to spend their Reputation points to alter the course of the story. There are fundamentally two different ways they can do this:
\begin{enumerate}
\item Spend one Reputation point and gain advantage on their next roll, or
\item Spend more than one Reputation point in order to fundamentally change the course of events so they align with the Reputation of their character.
\end{enumerate}

A player's Reputation is described by their Aspects. Players may only spend Reputation points when it is in a situation relevant to their Aspects.
In order to gain Reputation points after they have been spent, a player must play up their Flaw in a way that either hinders them, or is an otherwise exceptional example of role playing.

\paragraph{Combat}
(For a detailed explanation see \textbf{Combat} pg. \pageref{Combat})


Combat is a specific type of Opposed Contest. During combat, actions must take place over structured time. 

\paragraph{Wounds and Strain}
 (For a detailed explanation, see \textbf{Wounds and Strain} pg. \pageref{Wounds and Strain})


\begin{tcolorbox}
\begin{tabular}{ m{3cm} | p{10.5cm} }
\textbf{Injury Degree} & \textbf{Consequence} \\\hline


 Light	& No Penalty \\  \hline	
	
 Medium	& All contests rolled at -1 for each medium injury \\ \hline
	
 Severe	& All contests rolled at disadvantage \\ 
	 
\end{tabular}
\end{tcolorbox}

\begin{tcolorbox}[title = Treating Injuries]
\begin{tabular}{ m{3cm} | p{1cm}| p{3cm}| p{2cm}| p{3cm} }
\textbf{Severity} & \textbf{[TN]} & \textbf{Example} & \textbf{When Treated} & \textbf{Time to Heal} \\\hline


 Light	& 5 & Just a scratch. Sprained joint or pulled muscle & Removed & Instant \\  \hline	
	
 Medium	& 10 & Bruised or fractured bone. A profusely bleeding cut. & Bandaged & Days equal to ten minus Fitness rating \\ \hline
	
 Severe	& 15 & Badly broken or shattered bone. Bullet to the chest. & Bandaged & Days equal to fifteen minus Fitness rating \\ 
	 
\end{tabular}
\end{tcolorbox}

\paragraph{Initiative}
(For a detailed explanation see \textbf{Turn Order and Initiative} pg. \pageref{Turn Order})

2d6 + Awareness rating. Order of play proceeds from the character with the highest total to the character with the lowest total.

\paragraph{Movement}
(for a detailed explanation see \textbf{Movement} pg. \pageref{Movement})

1 Scene per movement action. 

Fitness Contest TN 8 to have an additional action. TN 15 to have a second additional action.

If you move more than one scene in a turn you cannot initiate a contest as well.

\textbf{Defending}
(For a detailed explanation see \textbf{Defending} pg. \pageref{Defending})

Counter-Attack - when you are being attacked you can attack back with your chosen Weapon or Rhetorical Strategy.

Evade - when you are attacked, you can roll 2d6 + Grit (Weapons) or Poise (Rhetorical Strategy) instead of Counter-Attacking to avoid the stakes.

\paragraph{Cover}
(For a detailed explanation see \textbf{Cover} pg. \pageref{Cover})

\begin{tcolorbox}
\begin{tabular}{ m{3cm} | p{5cm} | p{5cm} }
\textbf{Type of cover} & \textbf{Example}& \textbf{Effect} \\\hline


 Soft Cover	& Thin metal. A particularly robust window. & Decrease degree of wound severity once. \\  \hline	
	
 Hard Cover	& Reinforced wood. Stone. Layered sandbags & Decrease degree of wound severity twice \\ 
		 
\end{tabular}
\end{tcolorbox}

If an attacking character rolls doubles (a matching pair on any dice rolled), ignore the effects of cover.

\paragraph{Mounted Combat}
(For a detailed explanation, see \textbf{Mounted Combat} pg. \pageref{Mounted Combat})
\begin{itemize}
\item All Mounts have a shared Wounds and Strain track. Rhetorical Strategies cannot be used against mounts in combat. 
\item Once a Mount takes a severe Wound, they will suffer an additional light Wound whenever their rider takes a movement action.
\item Mounts can move up to three scenes per turn as their first movement action. Mounts do not receive their own turn during combat.
\item TN 10 Riding Contest to perform another movement action in a turn if riding.
\end{itemize}

\paragraph{Mount Attacks (Trampling)}
Riders who have traveled more than two scenes in a turn before attacking may gain Advantage against their target, or they may choose to Trample. If Trampling, treat the rider's Mount as the weapon, and use its qualities for for the attack.

If a character chooses to Trample, or their target is counterattacking with a weapon that has the Polearm quality, they do not gain Advantage.

\paragraph{Multiple Combat}
(Detailed explanation Multiple Participants In Combat pg. \pageref{Multiple Participants in Combat})
See the Section.

\paragraph{Death and Surrender}
(Detailed explanation Death and Surrender pg. \pageref{Death and Surrender})
See the Section.


\paragraph{Range Brackets}
(Detailed explanation Range pg. \pageref{Range})
\begin{itemize}
\item Short Range is within the same scene
\item Medium Range is from one to three scenes away
\item Long Range is from four to five scenes away
\item Very Long is more than five scenes away
\end{itemize}
Performing a Contest at a range longer than your weapon allows for incurs a -2 penalty for each extra range bracket