%----------------------------------------------------------------------------------------
%	CHAPTER 2
%----------------------------------------------------------------------------------------
\documentclass[../main.tex]{subfiles}

\chapterimage{Great_Plains_1x2.jpg} % Chapter heading image

\chapter{Landscapes and Locations}

\lettrine[lines=2]{L}{andscapes} are an iconic and important feature of the American West and its portrayals. The United States features varied landscapes, and Westerns, like Butch Cassidy and the Sundance Kid, show off the beauty of the West  from orchards to rocky plateaus and mountain vistas. The landscapes presented here are the ones featured around the hub city of Silver Springs, but doubtless there are many others, such as salt flats or oak savannas. Each of these landscapes have different benefits and challenges  associated with them.

\section{Landscapes}\label{Landscapes}


\begin{multicols}{2}

\paragraph{The Great Plains}
East of Silver Springs, the Great Plains stretch out to the edges of the horizon. It is an expanse of green (or golden) grass sometimes stubbly, grazed down to the earth by cattle, with occasional small copses of trees. Populating the open seas of grass are farmers, ranchers, nomadic Indians, bandits and outlaws, abandoned forts and more, all accessible so long as you're willing to cross the hundreds of miles of grassland to get there. A land of cattle drives, the last of the buffalo, and occasionally  train tracks, out here roads and company are luxuries, and cloudspotting is the national pastime. Towns and villages spring up at most crossroads, and while the work might be hard, the living is simple. On the prairie, time seems to slip away as every day looks a lot like the last. On the days that it doesn't, the weather may include sun, snow, rain, tornados, or all-out drought. An almanac of the Great Plains is cheap to acquire, but literacy is rare. God becomes a close friend out here, or at least someone who's good at listening when the urge to talk arises. What few topographical anomalies that exist can be seen for days before arrival, and serve as important landmarks. The absence of roads is rarely a problem; so long as you can keep traveling in the same direction you'll get somewhere eventually. Water and food aren't uncommon so long as you're comfortable hunting game, boiling water, and carrying everything else with you. Tobacco and coffee are worth their weight in gold on the plains, discussions of patience are moot, and orchards are the best mirages and the sweetest destinations. 

\paragraph{Great Plains Travel Difficulty}
\begin{description}
\item[Survival] 8
\item[Riding] 5
\item[Fitness] 7
\item[Awareness] 8
\end{description}



\paragraph{The Argent Range}
The Argent Range is a tall mountain range that extends down from Wyoming. Its peaks are always snowy, but it is known as being a relatively friendly range to navigate. There are a variety of passes that cross over it and a couple of nice valleys  for a traveler's respite. A couple of trains go through it as well, and it really feels like well-trafficked mountain range. The most dangerous thing on the mountains is not the weather, but the wolves that hunt in its  alpine forests. They don't tend to attack people, but with the pressure of hunters coming west they have started to take whatever they can get. The same can be said to be true of mountain lions. 
The Argent Mountains are the political divider between the states of Jefferson and Deseret. Before the mining rush, they had many names, but the term  argent  picked up amongst settlers  due to the massive amount of silver being pulled out of the ground in the area. Other than mining, and the occasional fur trapper, most people in and around the mountains are traveling across them. In all seasons, the mountains are prone to storms and quick temperature changes, but they are especially treacherous in winter, when the snow flies. The Indians hold certain peaks of the mountain to be sacred, and they feature prominently in their stories. They do not take well to the railroad companies modifying the landscape.

\paragraph{Argent Range Travel Difficulty}

\begin{description}
\item[Survival] 15
\item[Riding] 9
\item[Fitness] 7 (+4 without suitable equipment)
\item[Awareness] 16
\end{description}


\paragraph{Chacahuela Desert }

The Chacahuela Desert, or the Chuckwalla to most settlers, is a large desert that borders the Great Plains to the North and the Argent Mountains in the West. Because of a rain shadow cast by the Argents, the Chuckwalla rarely sees moisture of any kind. Named for the large lizards that populate the rocky scrublands, Chuckwalla lizards are eaten by the natives and considered a delicacy. To non-locals, their incredibly rich yellow meat is filling and not unpalatable. The Chuckwalla is a place of rocks, sand and cacti, but do not think it isn't a diverse place. From high places you can see that it is a sprawling and varied landscape composed of rocky hills, sandy and dusty plains, dry streambeds in ravines and petrified logs demarcating where ancient forests used to stand. Throughout the landscape are sun bleached bones, and occasionally the unpredictable winds turn over rocks to reveal small oceanic fossils. A few strong plants cling to the shade, near streambed habitats, and they do produce flowers and berries  from time to time. It may not be an especially hospitable place, but it is beautiful if you can withstand the heat and thirst. But do take care: what little rain that falls causes severe flash floods every one to three years, wreaking devastation on the unprepared. It is said that the rains leave in their wake a momentary edenic landscape of wildflowers, butterflies, frogs, and pleasant aromas. The Chuckwalla is a place where someone might go to seek out Indians, for better or for worse, to witness something seemingly divine at the right time of year, or to find meaning in the wastes. The desert, after all, is a simple place with few secrets.

\paragraph{Chacahuela Desert Travel Difficulty}

\begin{description}
\item[Survival] 14
\item[Riding] 8
\item[Fitness] 12
\item[Awareness] 15
\end{description}




\paragraph{The Plateau Classical}

The Plateau Classical, named by French explorer Guy LeBlanc, more or less means the rock plateau, which to a certain extent is what this plateau actually is. Located in the Chacahuela Desert, this monolithic yellow-brown stone  structure suddenly rises up from the ground, casting shade all around it, on nearly every side. Water erosion from heavy rains has worn channels around its edges , and silt from the top of the plateau seems to kick up into the air and into the eyes, noses, and mouths of unprepared travelers. The base of the rock is surprisingly green, and at the bottom of every waterfall ravine, there are some plants or at least moss that has gathered there. The base of this plateau is filled with  camping grounds used commonly by nomadic Indians, as walking around the base is the most comfortable way of crossing the middle of the desert. There are a few paths to the top of the plateau if you know where to look and riding atop it is the fastest and most direct way across the middle of the desert, if the lack of shelter doesn't get to you before you make it. The wind is as fierce as the sun and it's dangerous to cross in nearly all seasons. The top of the plateau has no shelter, anyone on the plateau can see the length of it, and that means crossing it in anything less than twelve hours is not recommended by anyone because  it burns during the day and freezes at night. 

\paragraph{The Plateau Classical Travel Difficulty}

\begin{description}
\item[Survival] 12
\item[Riding] 12 (+4 if towing wagon or coach)
\item[Fitness] 16 (+4 without suitable equipment)
\item[Awareness] 10
\end{description}



\paragraph{Jefferson Stripe River Canyon}

The Jefferson Stripe is a massive river that runs out of the Argent Range and continues southeast until it joins the Jefferson-Turage River. Over the course of its vast length, the river goes through stretches of waterfall, pooling, whitewater, calm, wide and shallow, and deep and narrow sections. Those that know it well can always tell how close they are to its origin as the water gets greener closer to the mountains and browner closer  to the Mississippi. Silver Springs was originally built on its banks, and the river still borders the whole of the southwestern part of town. That city stretch is dominated by slow moving pools and a somewhat foul smell. The river divides runs through the the Great Plains to the East for most of its journey until it joins the Jefferson-Turage river on the Jefferson Turage border.
A couple miles past the city, the Stripe flows through the famously deep Jefferson Stripe Canyon until that canyon gradually shortens and fades away as it crosses the flat stretches of the Midwest States. The canyon at its height is easily over a hundred feet deep, and it reaches this depth just East of the city at the magnificent Jefferson Falls. Due to it's proximity to the city and the rail lines it's become a happy hiding spot for a variety of bandit groups, who have explored nearly all of the game trails that navigate the Canyon's rocky walls. The cave complexes within the walls of the Canyon provide good impromptu hideouts, and the variety of nooks, crannies, spaces between boulders and other secret spots provide great places to hide anything from gold to stolen goods. There are very few places to cross the Canyon, most of them small rickety bridges, except for the JDSF train bridge that splits off from the main line a couple miles east of town and the bridge at Carolina's Crossing. Carolina runs an inn right next to the only sturdy well engineered bridge, about two coaches wide and 500 yards across. 

\paragraph{Jefferson Stripe River Canyon Travel Difficulty}

\begin{description}
\item[Survival] 6
\item [Riding] 6 (+4 if towing wagon/ coach)
\item[Fitness] 6 (+4 without suitable equipment)
\item[Awareness] 18
\end{description}



\paragraph{Centerstone Hills}

The Centerstone Hills aren't named for their geographic placement, but for their spiritual importance to local tribes. A unique landscape found in the deepest parts of the Great Plains, the Centerstone Hills can be seen from miles away as dark features rising up over the horizon. Forested with hardy pine and spruce trees, and interspaced with both lush meadows and monolithic granite formations, it's no wonder that the Centerstone Hills were sacred to virtually every Native group that came across them. 
The Centerstone Hills are one of the few places in the shrinking frontier that is entirely ceded to the Five Nations, owing to the difficulty of farming its rocky, vegetation-blanketed hills. Until recently the Centerstone Hills were promised to exclusively the Five Nations in the Five Nations Treaty  by the federal government. The Centerstone Reservation is to some a haven to avoid conflict with the settlers, and to adapt to a  new way of life, while to others it is a symbol of the complacency of the Nations and a sign of a defeat. More recently, however, rumors of gold in the hills have piqued the interest of settlers, and there are rumors and rumblings that the cries of the prospectors to shrink or close the reservation are gaining traction. Gold seekers are already moving into the area and tensions are rising faster than they ever have. The disrespect of the sacred places and the feeling of betrayal some Nations' Indians feel at this turn of events has brought back the anger that the Nations strove to avoid in accepting the Treaty. 

\paragraph{Centerstone Hills Travel Difficulty}

\begin{description}
\item[Survival] 12
\item[Riding] 8
\item[Fitness] 8
\item[Awareness] 16
\end{description}


\paragraph{Ghost Mountains}

The Ghost Mountains are a small, isolated range of mountains located significantly East of Silver Springs. While the mountains themselves are unmapped and scarcely populated, the foothills at their base are home to a community of prospectors, hermits, and more permanent townsfolk. The mountains rise up in a very odd way out of the Centerstone Hills. They rise just high enough to have snowy peaks, and the odd formation seems to almost be made up of rocks spiraling around each other. Because of their odd structure, there are numerous paths up, around and through the mountains, although the last thousand feet on the highest peaks seem to have no easy access and require climbing gear to scale.
Most of the people nearby make a living working as guides and outfitters, or try their luck at prospecting for gold or silver, while a few seek out those who've become lost in the mountains for pay- whether they want to be found or not. The Ghost Mountain community doesn't ostracize Indians to the extent that most settlers do, and it's for this reason that many travelers avoid the supposedly cursed range, though for many that's just a way of putting a label on the nameless fear that seems to lurk there. The Ghost Mountains have a strange mythos associated with them. Tales of lost mines spilling over with gold and silver, fortune-telling mystics, and hidden entrances leading to the homes of impossibly beautiful, impossibly wise people living under the rocky cliffs abound. The Ghost Mountains are one of the most Westerly places not fully mapped, and even the most seasoned veteran guides regularly find new canyons, rock formations and petroglyphs. It is a strange place, home to plants and animals that can't be found anywhere else, where landmarks seem to move from expedition to expedition. The townspeople actively dislike and try to avoid surveyors from the area, as if somehow they'd lose something if it were to be mapped and catalogued. A place of strangeness to be sure, where desperate cowboys can vanish, never to be seen again, and lucky treasure hunters might strike it rich!

\paragraph{Ghost Mountains Travel Difficulty}

\begin{description}
\item[Survival] 12
\item[Riding] 18 (+4 if towing wagon/ coach)
\item[Fitness] 14 (+4 without suitable equipment)
\item[Awareness] 14
\end{description}

\end{multicols}


\section{Settlements}

	The settlements of the American West are an important part of the landscape as well. While they don't occupy as much space, they have lots of character. Presented in this section (briefly) are Silver Springs, which is the central  city of the Jefferson Territory, as well as examples of other settlements across the expanse of the West. There are many different kinds of possible settlements, many of which do not conform to the examples we've given here, and you should feel free to imagine other kinds of settlements, or combine the types of settlements we've presented below. 

\begin{multicols}{2}


\paragraph{Silver Springs}
	Silver Springs is the capital  city of Jefferson. It has the biggest train station, the largest number of people of all kinds, meeting and mingling, and the most challenges concerning growth and law. Silver Springs is host to most of the big names in the state, possibly even the West, and everyone who is anyone has been there. It's where the most goods are shipped to and from, and ultimately any experience that can be found in the West is or will be there.
	
\paragraph{Silver Springs Travel Difficulty}

\begin{description}
\item[Survival] 6
\item[Riding] 5
\item[Fitness] 4
\item[Awareness] 10 (+3 if without a map)
\end{description}

\paragraph{A Smaller Settlement}

To the people of Silver Springs the rest of the frontier seems empty, but in reality the West is dotted with smaller settlements. It's places like these where your reputation matters the most. People out here will be exposed to stories told by travelers,  but without the cynicism bred by urban life, they will take them a lot more seriously and literally.  The smallest example of a settlement you might find would be a single-family homestead. Larger settlements might have things like a general store, an inn, stables, or a clinic. Most places have a workshop of some kind and many have a practicing craftsman like a blacksmith.

Languages Spoken: Various (English, Native, Spanish, French, etc.)

\paragraph{Smaller Settlement Travel Difficulty}

\begin{description}
\item[Survival] 6
\item[Riding] 4
\item[Fitness] 4
\item[Awareness] 7 (+2 if without a map)
\end{description}


\paragraph{A Hopeful Settlement}

People in settlements like this are often people displaced by the war, and they are frequently more trusting of people who fought on the same side they did. Some small settlements are also of a group of people who immigrated from a specific country, like Poland or Denmark.  These people probably remember and practice the customs of their place of origin, speak their native  language, perhaps better than english, and generally vary wildly between being hospitable and wary. By and large American settlers tend to distrust Natives and Enochites but are not necessarily in a position to refuse business. Generally these settlements host a mix of different people, some willing to help a traveler in need, others who distrust outsiders. 

Languages Spoken: Various Immigrant Languages (English, Spanish, French, Polish, Danish, German, etc.)

\paragraph{Hopeful Settlement Travel Difficulty}

\begin{description}
\item[Survival] 4
\item[Riding] 4
\item[Fitness] 4
\item[Awareness] (+2 if without a map)
\end{description}


\paragraph{A Mexican Village}

The Mexican American communities are more plentiful in the southern reaches of Jefferson, and while this far north there aren't remnants of Mexican towns, most Mexican immigrants are still more comfortable living together then amongst the xenophobic townspeople of Silver Springs. Mexican Villages are some of the most unconditionally welcoming places in all of Jefferson, although any slight against the community often moves to quick ostracization. They are usually friendly towards Indians provided that those Indians are friendly towards them, and so Indian heritage may have made its way into the families in these towns. These communities are usually agrarian and catholic, and they generously share what they have with travelers in a good or bad season. They are usually willing to barter for gold, supplies, tools, and alcohol. Most villages feature a range of craftspeople and possibly an inn or a saloon. As mentioned above, however, they are incredibly insular and theft is noticed very quickly. These communities feel warm and content, with nightly singing and dancing when the weather is nice. 

Languages Spoken: Spanish (Mexican Dialect), Native, Various Immigrant Languages.

\paragraph{Mexican Village Travel Difficulty}

\begin{description}
\item[Survival] 5
\item[Riding] 5
\item[Fitness] 5
\item[Awareness] 6 (+4 if without a map)
\end{description}


\paragraph{A Ghost Town }

	Ghost towns were places at one point or another; often mining, logging, or trapping camps set up by wild entrepreneurs with a few permanent buildings such as log houses or common rooms. When the reason for setting up this community disappeared, or danger lurked,  people left. Now many ghost towns still have a running inn, and buildings in various states of disrepair, especially if they are close enough to a crossroads to be useful. Occasionally people looking to hide out might take up rooms there, bandits might use them as a hideout, or somebody hoping to find gold in the hills after everyone else had given up hope. Regardless of who is there, Ghost Towns are only home to the desperate and should evoke a wary attitude in passers by.

Languages Spoken: Various Immigrant Languages

\paragraph{Ghost Town Travel Difficulty}

\begin{description}
\item[Survival] 12
\item[Riding] 6
\item[Fitness] 6
\item[Awareness] 1 (+2 if without a map)
\end{description}


\paragraph{An Enochite Compound }

Enochite compounds are abandoned US Army forts, homesteads, and even some small villages that have been transformed into armed holdouts against the Federal Government. These compounds are usually full of welcoming people interested in converting others to their faith and lifestyle. Some of them are havens for Enochites hoping to continue controversial practices, such as polygamy, which many reject, while others are forward thinking and trying to adapt to the new state of affairs where the Church is concerned. Those who are more religious are willing to host and feed travelers overnight out of their goodwill, while other compounds driven by their conditions might be less willing to share what little they have with non-believers. 
The mainstream Church of Enoch has reached an uneasy truce with the Federal government after disastrous conflicts in the years before the Civil War, although many felt that Church President Theron Marsh's decision to cease violence against the Federal government was a mistake. If you are in any way associated with the federal government and plan on going in and amongst the Enochites it may be best to keep your affiliations hidden.

Languages Spoken: English

\paragraph{Enochite Compound Travel Difficulty}

\begin{description}
\item[Survival] 6
\item[Riding] 4
\item[Fitness] 4
\item[Awareness] 8 (+2 if without a map)
\end{description}

\paragraph{A Civilized Five Reservation}
	The Civilized Five live in poverty and squalor with little hospitality to offer travelers.  Their reservations are enclosed with barbed-wire fences, and are little more than a dozen barracks. These reservations are cramped, and disease spreads quickly within their boundaries. Death is common and new corpses are found every week. The reasons are plenty: suicide, alcoholism, and sickness, to name a few. Reservations are usually guarded by a small contingent of the U.S. Army, with each contingent numbering about fifty men. These troops are stationed to prevent revolts, but they frequently agitate and abuse their captives. 

Languages Spoken: Native, English

\paragraph{Reservation Travel Difficulty}

\begin{description}
\item[Survival] 14
\item[Riding] 6
\item[Fitness] 6
\item[Awareness] 10
\end{description}

\paragraph{A Tribal Encampment}
	When the Confederacy of the Hawk migrated to the lands east of Argent Mountain, they put a massive strain on the natural produce. Within a year, the land was barren, and new crops were slow to grow. Because of this, Kenu leads raids against the railways in Jefferson State, of which almost all have failed. Because victory is scarce, these camps prize whatever spoils of war can be found. They dangle scalps from their horse's braids and ride into battle bedecked in their loot: guns, uniforms, and other American militaria are highly prized. Their warring, nomadic existence means that these camps are temporary, and are little more than a place of respite until the next campaign. 

Languages Spoken: Native only

\paragraph{Tribal Encamplent Travel Difficulty}

\begin{description}
\item[Survival] 14
\item[Riding] 8
\item[Fitness] 8
\item[Awareness] 12
\end{description}

\end{multicols}